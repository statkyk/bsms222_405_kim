% Options for packages loaded elsewhere
\PassOptionsToPackage{unicode}{hyperref}
\PassOptionsToPackage{hyphens}{url}
%
\documentclass[
]{article}
\usepackage{amsmath,amssymb}
\usepackage{lmodern}
\usepackage{ifxetex,ifluatex}
\ifnum 0\ifxetex 1\fi\ifluatex 1\fi=0 % if pdftex
  \usepackage[T1]{fontenc}
  \usepackage[utf8]{inputenc}
  \usepackage{textcomp} % provide euro and other symbols
\else % if luatex or xetex
  \usepackage{unicode-math}
  \defaultfontfeatures{Scale=MatchLowercase}
  \defaultfontfeatures[\rmfamily]{Ligatures=TeX,Scale=1}
\fi
% Use upquote if available, for straight quotes in verbatim environments
\IfFileExists{upquote.sty}{\usepackage{upquote}}{}
\IfFileExists{microtype.sty}{% use microtype if available
  \usepackage[]{microtype}
  \UseMicrotypeSet[protrusion]{basicmath} % disable protrusion for tt fonts
}{}
\makeatletter
\@ifundefined{KOMAClassName}{% if non-KOMA class
  \IfFileExists{parskip.sty}{%
    \usepackage{parskip}
  }{% else
    \setlength{\parindent}{0pt}
    \setlength{\parskip}{6pt plus 2pt minus 1pt}}
}{% if KOMA class
  \KOMAoptions{parskip=half}}
\makeatother
\usepackage{xcolor}
\IfFileExists{xurl.sty}{\usepackage{xurl}}{} % add URL line breaks if available
\IfFileExists{bookmark.sty}{\usepackage{bookmark}}{\usepackage{hyperref}}
\hypersetup{
  pdftitle={Chapter 2},
  hidelinks,
  pdfcreator={LaTeX via pandoc}}
\urlstyle{same} % disable monospaced font for URLs
\usepackage[margin=1in]{geometry}
\usepackage{color}
\usepackage{fancyvrb}
\newcommand{\VerbBar}{|}
\newcommand{\VERB}{\Verb[commandchars=\\\{\}]}
\DefineVerbatimEnvironment{Highlighting}{Verbatim}{commandchars=\\\{\}}
% Add ',fontsize=\small' for more characters per line
\usepackage{framed}
\definecolor{shadecolor}{RGB}{248,248,248}
\newenvironment{Shaded}{\begin{snugshade}}{\end{snugshade}}
\newcommand{\AlertTok}[1]{\textcolor[rgb]{0.94,0.16,0.16}{#1}}
\newcommand{\AnnotationTok}[1]{\textcolor[rgb]{0.56,0.35,0.01}{\textbf{\textit{#1}}}}
\newcommand{\AttributeTok}[1]{\textcolor[rgb]{0.77,0.63,0.00}{#1}}
\newcommand{\BaseNTok}[1]{\textcolor[rgb]{0.00,0.00,0.81}{#1}}
\newcommand{\BuiltInTok}[1]{#1}
\newcommand{\CharTok}[1]{\textcolor[rgb]{0.31,0.60,0.02}{#1}}
\newcommand{\CommentTok}[1]{\textcolor[rgb]{0.56,0.35,0.01}{\textit{#1}}}
\newcommand{\CommentVarTok}[1]{\textcolor[rgb]{0.56,0.35,0.01}{\textbf{\textit{#1}}}}
\newcommand{\ConstantTok}[1]{\textcolor[rgb]{0.00,0.00,0.00}{#1}}
\newcommand{\ControlFlowTok}[1]{\textcolor[rgb]{0.13,0.29,0.53}{\textbf{#1}}}
\newcommand{\DataTypeTok}[1]{\textcolor[rgb]{0.13,0.29,0.53}{#1}}
\newcommand{\DecValTok}[1]{\textcolor[rgb]{0.00,0.00,0.81}{#1}}
\newcommand{\DocumentationTok}[1]{\textcolor[rgb]{0.56,0.35,0.01}{\textbf{\textit{#1}}}}
\newcommand{\ErrorTok}[1]{\textcolor[rgb]{0.64,0.00,0.00}{\textbf{#1}}}
\newcommand{\ExtensionTok}[1]{#1}
\newcommand{\FloatTok}[1]{\textcolor[rgb]{0.00,0.00,0.81}{#1}}
\newcommand{\FunctionTok}[1]{\textcolor[rgb]{0.00,0.00,0.00}{#1}}
\newcommand{\ImportTok}[1]{#1}
\newcommand{\InformationTok}[1]{\textcolor[rgb]{0.56,0.35,0.01}{\textbf{\textit{#1}}}}
\newcommand{\KeywordTok}[1]{\textcolor[rgb]{0.13,0.29,0.53}{\textbf{#1}}}
\newcommand{\NormalTok}[1]{#1}
\newcommand{\OperatorTok}[1]{\textcolor[rgb]{0.81,0.36,0.00}{\textbf{#1}}}
\newcommand{\OtherTok}[1]{\textcolor[rgb]{0.56,0.35,0.01}{#1}}
\newcommand{\PreprocessorTok}[1]{\textcolor[rgb]{0.56,0.35,0.01}{\textit{#1}}}
\newcommand{\RegionMarkerTok}[1]{#1}
\newcommand{\SpecialCharTok}[1]{\textcolor[rgb]{0.00,0.00,0.00}{#1}}
\newcommand{\SpecialStringTok}[1]{\textcolor[rgb]{0.31,0.60,0.02}{#1}}
\newcommand{\StringTok}[1]{\textcolor[rgb]{0.31,0.60,0.02}{#1}}
\newcommand{\VariableTok}[1]{\textcolor[rgb]{0.00,0.00,0.00}{#1}}
\newcommand{\VerbatimStringTok}[1]{\textcolor[rgb]{0.31,0.60,0.02}{#1}}
\newcommand{\WarningTok}[1]{\textcolor[rgb]{0.56,0.35,0.01}{\textbf{\textit{#1}}}}
\usepackage{graphicx}
\makeatletter
\def\maxwidth{\ifdim\Gin@nat@width>\linewidth\linewidth\else\Gin@nat@width\fi}
\def\maxheight{\ifdim\Gin@nat@height>\textheight\textheight\else\Gin@nat@height\fi}
\makeatother
% Scale images if necessary, so that they will not overflow the page
% margins by default, and it is still possible to overwrite the defaults
% using explicit options in \includegraphics[width, height, ...]{}
\setkeys{Gin}{width=\maxwidth,height=\maxheight,keepaspectratio}
% Set default figure placement to htbp
\makeatletter
\def\fps@figure{htbp}
\makeatother
\setlength{\emergencystretch}{3em} % prevent overfull lines
\providecommand{\tightlist}{%
  \setlength{\itemsep}{0pt}\setlength{\parskip}{0pt}}
\setcounter{secnumdepth}{-\maxdimen} % remove section numbering
\ifluatex
  \usepackage{selnolig}  % disable illegal ligatures
\fi

\title{Chapter 2}
\author{}
\date{\vspace{-2.5em}}

\begin{document}
\maketitle

\hypertarget{chapter-2-r-basics}{%
\section{Chapter 2 R basics}\label{chapter-2-r-basics}}

In this book, we will be using the R software environment for all our
analysis. You will learn R and data analysis techniques simultaneously.
To follow along you will therefore need access to R. We also recommend
the use of an integrated development environment (IDE), such as RStudio,
to save your work. Note that it is common for a course or workshop to
offer access to an R environment and an IDE through your web browser, as
done by RStudio cloud12. If you have access to such a resource, you
don't need to install R and RStudio. However, if you intend on becoming
an advanced data analyst, we highly recommend installing these tools on
your computer13. Both R and RStudio are free and available online.

\hypertarget{case-study-us-gun-murders}{%
\subsection{2.1 Case study: US Gun
Murders}\label{case-study-us-gun-murders}}

Imagine you live in Europe and are offered a job in a US company with
many locations across all states. It is a great job, but news with
headlines such as US Gun Homicide Rate Higher Than Other Developed
Countries14 have you worried. Charts like this may concern you even
more:

Or even worse, this version from everytown.org:

But then you remember that the US is a large and diverse country with 50
very different states as well as the District of Columbia (DC).

California, for example, has a larger population than Canada, and 20 US
states have populations larger than that of Norway. In some respects,
the variability across states in the US is akin to the variability
across countries in Europe. Furthermore, although not included in the
charts above, the murder rates in Lithuania, Ukraine, and Russia are
higher than 4 per 100,000. So perhaps the news reports that worried you
are too superficial. You have options of where to live and want to
determine the safety of each particular state. We will gain some
insights by examining data related to gun homicides in the US during
2010 using R.

Before we get started with our example, we need to cover logistics as
well as some of the very basic building blocks that are required to gain
more advanced R skills. Be aware that the usefulness of some of these
building blocks may not be immediately obvious, but later in the book
you will appreciate having mastered these skills.

\hypertarget{the-very-basics}{%
\subsection{2.2 The very basics}\label{the-very-basics}}

Before we get started with the motivating dataset, we need to cover the
very basics of R.

\hypertarget{objects}{%
\subsubsection{2.2.1 Objects}\label{objects}}

Suppose a high school student asks us for help solving several quadratic
equations of the form ax\^{}2+bx+c = 0. The quadratic formula gives us
the solutions:

which of course change depending on the values of a, b, and c.~One
advantage of programming languages is that we can define variables and
write expressions with these variables, similar to how we do so in math,
but obtain a numeric solution. We will write out general code for the
quadratic equation below, but if we are asked to solve x\^{}2 + x -1 =
0, then we define:

\begin{Shaded}
\begin{Highlighting}[]
\NormalTok{a }\OtherTok{\textless{}{-}} \DecValTok{1}
\NormalTok{b }\OtherTok{\textless{}{-}} \DecValTok{1}
\NormalTok{c }\OtherTok{\textless{}{-}} \SpecialCharTok{{-}}\DecValTok{1}
\end{Highlighting}
\end{Shaded}

which stores the values for later use. We use \textless- to assign
values to the variables.

We can also assign values using = instead of \textless-, but we
recommend against using = to avoid confusion.

Copy and paste the code above into your console to define the three
variables. Note that R does not print anything when we make this
assignment. This means the objects were defined successfully. Had you
made a mistake, you would have received an error message.

To see the value stored in a variable, we simply ask R to evaluate a and
it shows the stored value:

\begin{Shaded}
\begin{Highlighting}[]
\NormalTok{a}
\end{Highlighting}
\end{Shaded}

\begin{verbatim}
## [1] 1
\end{verbatim}

A more explicit way to ask R to show us the value stored in a is using
print like this:

\begin{Shaded}
\begin{Highlighting}[]
\FunctionTok{print}\NormalTok{(a)}
\end{Highlighting}
\end{Shaded}

\begin{verbatim}
## [1] 1
\end{verbatim}

We use the term object to describe stuff that is stored in R. Variables
are examples, but objects can also be more complicated entities such as
functions, which are described later.

\hypertarget{the-workspace}{%
\subsection{2.2.2 The workspace}\label{the-workspace}}

As we define objects in the console, we are actually changing the
workspace. You can see all the variables saved in your workspace by
typing:

\begin{Shaded}
\begin{Highlighting}[]
\FunctionTok{ls}\NormalTok{()}
\end{Highlighting}
\end{Shaded}

\begin{verbatim}
## [1] "a" "b" "c"
\end{verbatim}

In RStudio, the Environment tab shows the values:

We should see a, b, and c.~If you try to recover the value of a variable
that is not in your workspace, you receive an error. For example, if you
type x you will receive the following message:

Error: object `x' not found.

Now since these values are saved in variables, to obtain a solution to
our equation, we use the quadratic formula:

\begin{Shaded}
\begin{Highlighting}[]
\NormalTok{(}\SpecialCharTok{{-}}\NormalTok{b }\SpecialCharTok{+} \FunctionTok{sqrt}\NormalTok{(b}\SpecialCharTok{\^{}}\DecValTok{2} \SpecialCharTok{{-}} \DecValTok{4}\SpecialCharTok{*}\NormalTok{a}\SpecialCharTok{*}\NormalTok{c) ) }\SpecialCharTok{/}\NormalTok{ ( }\DecValTok{2}\SpecialCharTok{*}\NormalTok{a )}
\end{Highlighting}
\end{Shaded}

\begin{verbatim}
## [1] 0.618034
\end{verbatim}

\begin{Shaded}
\begin{Highlighting}[]
\NormalTok{(}\SpecialCharTok{{-}}\NormalTok{b }\SpecialCharTok{{-}} \FunctionTok{sqrt}\NormalTok{(b}\SpecialCharTok{\^{}}\DecValTok{2} \SpecialCharTok{{-}} \DecValTok{4}\SpecialCharTok{*}\NormalTok{a}\SpecialCharTok{*}\NormalTok{c) ) }\SpecialCharTok{/}\NormalTok{ ( }\DecValTok{2}\SpecialCharTok{*}\NormalTok{a )}
\end{Highlighting}
\end{Shaded}

\begin{verbatim}
## [1] -1.618034
\end{verbatim}

\hypertarget{functions}{%
\subsubsection{2.2.3 Functions}\label{functions}}

Once you define variables, the data analysis process can usually be
described as a series of functions applied to the data. R includes
several predefined functions and most of the analysis pipelines we
construct make extensive use of these.

We already used the install.packages, library, and ls functions. We also
used the function sqrt to solve the quadratic equation above. There are
many more prebuilt functions and even more can be added through
packages. These functions do not appear in the workspace because you did
not define them, but they are available for immediate use.

In general, we need to use parentheses to evaluate a function. If you
type ls, the function is not evaluated and instead R shows you the code
that defines the function. If you type ls() the function is evaluated
and, as seen above, we see objects in the workspace.

Unlike ls, most functions require one or more arguments. Below is an
example of how we assign an object to the argument of the function log.
Remember that we earlier defined a to be 1:

\begin{Shaded}
\begin{Highlighting}[]
\FunctionTok{log}\NormalTok{(}\DecValTok{8}\NormalTok{)}
\end{Highlighting}
\end{Shaded}

\begin{verbatim}
## [1] 2.079442
\end{verbatim}

\begin{Shaded}
\begin{Highlighting}[]
\FunctionTok{log}\NormalTok{(a)}
\end{Highlighting}
\end{Shaded}

\begin{verbatim}
## [1] 0
\end{verbatim}

You can find out what the function expects and what it does by reviewing
the very useful manuals included in R. You can get help by using the
help function like this:

\begin{Shaded}
\begin{Highlighting}[]
\FunctionTok{help}\NormalTok{(}\StringTok{"log"}\NormalTok{)}
\end{Highlighting}
\end{Shaded}

\begin{verbatim}
## starting httpd help server ... done
\end{verbatim}

For most functions, we can also use this shorthand:

\begin{Shaded}
\begin{Highlighting}[]
\NormalTok{?log}
\end{Highlighting}
\end{Shaded}

The help page will show you what arguments the function is expecting.
For example, log needs x and base to run. However, some arguments are
required and others are optional. You can determine which arguments are
optional by noting in the help document that a default value is assigned
with =. Defining these is optional. For example, the base of the
function log defaults to base = exp(1) making log the natural log by
default.

If you want a quick look at the arguments without opening the help
system, you can type:

\begin{Shaded}
\begin{Highlighting}[]
\FunctionTok{args}\NormalTok{(log)}
\end{Highlighting}
\end{Shaded}

\begin{verbatim}
## function (x, base = exp(1)) 
## NULL
\end{verbatim}

You can change the default values by simply assigning another object:

\begin{Shaded}
\begin{Highlighting}[]
\FunctionTok{log}\NormalTok{(}\DecValTok{8}\NormalTok{, }\AttributeTok{base =} \DecValTok{2}\NormalTok{)}
\end{Highlighting}
\end{Shaded}

\begin{verbatim}
## [1] 3
\end{verbatim}

Note that we have not been specifying the argument x as such:

\begin{Shaded}
\begin{Highlighting}[]
\FunctionTok{log}\NormalTok{(}\AttributeTok{x =} \DecValTok{8}\NormalTok{, }\AttributeTok{base =} \DecValTok{2}\NormalTok{)}
\end{Highlighting}
\end{Shaded}

\begin{verbatim}
## [1] 3
\end{verbatim}

The above code works, but we can save ourselves some typing: if no
argument name is used, R assumes you are entering arguments in the order
shown in the help file or by args. So by not using the names, it assumes
the arguments are x followed by base:

\begin{Shaded}
\begin{Highlighting}[]
\FunctionTok{log}\NormalTok{(}\DecValTok{8}\NormalTok{,}\DecValTok{2}\NormalTok{)}
\end{Highlighting}
\end{Shaded}

\begin{verbatim}
## [1] 3
\end{verbatim}

If using the arguments' names, then we can include them in whatever
order we want:

\begin{Shaded}
\begin{Highlighting}[]
\FunctionTok{log}\NormalTok{(}\AttributeTok{base =} \DecValTok{2}\NormalTok{, }\AttributeTok{x =} \DecValTok{8}\NormalTok{)}
\end{Highlighting}
\end{Shaded}

\begin{verbatim}
## [1] 3
\end{verbatim}

To specify arguments, we must use =, and cannot use \textless-.

There are some exceptions to the rule that functions need the
parentheses to be evaluated. Among these, the most commonly used are the
arithmetic and relational operators. For example:

\begin{Shaded}
\begin{Highlighting}[]
\DecValTok{2} \SpecialCharTok{\^{}} \DecValTok{3}
\end{Highlighting}
\end{Shaded}

\begin{verbatim}
## [1] 8
\end{verbatim}

You can see the arithmetic operators by typing:

\begin{Shaded}
\begin{Highlighting}[]
\FunctionTok{help}\NormalTok{(}\StringTok{"+"}\NormalTok{)}
\end{Highlighting}
\end{Shaded}

or

\begin{Shaded}
\begin{Highlighting}[]
\NormalTok{?}\StringTok{"+"}
\end{Highlighting}
\end{Shaded}

and the relational operators by typing:

\begin{Shaded}
\begin{Highlighting}[]
\FunctionTok{help}\NormalTok{(}\StringTok{"\textgreater{}"}\NormalTok{)}
\end{Highlighting}
\end{Shaded}

or

\begin{Shaded}
\begin{Highlighting}[]
\NormalTok{?}\StringTok{"\textgreater{}"}
\end{Highlighting}
\end{Shaded}

\hypertarget{other-prebuilt-objects}{%
\subsubsection{2.2.4 Other prebuilt
objects}\label{other-prebuilt-objects}}

There are several datasets that are included for users to practice and
test out functions. You can see all the available datasets by typing:

\begin{Shaded}
\begin{Highlighting}[]
\FunctionTok{data}\NormalTok{()}
\end{Highlighting}
\end{Shaded}

This shows you the object name for these datasets. These datasets are
objects that can be used by simply typing the name. For example, if you
type:

\begin{Shaded}
\begin{Highlighting}[]
\NormalTok{co2}
\end{Highlighting}
\end{Shaded}

\begin{verbatim}
##         Jan    Feb    Mar    Apr    May    Jun    Jul    Aug    Sep    Oct
## 1959 315.42 316.31 316.50 317.56 318.13 318.00 316.39 314.65 313.68 313.18
## 1960 316.27 316.81 317.42 318.87 319.87 319.43 318.01 315.74 314.00 313.68
## 1961 316.73 317.54 318.38 319.31 320.42 319.61 318.42 316.63 314.83 315.16
## 1962 317.78 318.40 319.53 320.42 320.85 320.45 319.45 317.25 316.11 315.27
## 1963 318.58 318.92 319.70 321.22 322.08 321.31 319.58 317.61 316.05 315.83
## 1964 319.41 320.07 320.74 321.40 322.06 321.73 320.27 318.54 316.54 316.71
## 1965 319.27 320.28 320.73 321.97 322.00 321.71 321.05 318.71 317.66 317.14
## 1966 320.46 321.43 322.23 323.54 323.91 323.59 322.24 320.20 318.48 317.94
## 1967 322.17 322.34 322.88 324.25 324.83 323.93 322.38 320.76 319.10 319.24
## 1968 322.40 322.99 323.73 324.86 325.40 325.20 323.98 321.95 320.18 320.09
## 1969 323.83 324.26 325.47 326.50 327.21 326.54 325.72 323.50 322.22 321.62
## 1970 324.89 325.82 326.77 327.97 327.91 327.50 326.18 324.53 322.93 322.90
## 1971 326.01 326.51 327.01 327.62 328.76 328.40 327.20 325.27 323.20 323.40
## 1972 326.60 327.47 327.58 329.56 329.90 328.92 327.88 326.16 324.68 325.04
## 1973 328.37 329.40 330.14 331.33 332.31 331.90 330.70 329.15 327.35 327.02
## 1974 329.18 330.55 331.32 332.48 332.92 332.08 331.01 329.23 327.27 327.21
## 1975 330.23 331.25 331.87 333.14 333.80 333.43 331.73 329.90 328.40 328.17
## 1976 331.58 332.39 333.33 334.41 334.71 334.17 332.89 330.77 329.14 328.78
## 1977 332.75 333.24 334.53 335.90 336.57 336.10 334.76 332.59 331.42 330.98
## 1978 334.80 335.22 336.47 337.59 337.84 337.72 336.37 334.51 332.60 332.38
## 1979 336.05 336.59 337.79 338.71 339.30 339.12 337.56 335.92 333.75 333.70
## 1980 337.84 338.19 339.91 340.60 341.29 341.00 339.39 337.43 335.72 335.84
## 1981 339.06 340.30 341.21 342.33 342.74 342.08 340.32 338.26 336.52 336.68
## 1982 340.57 341.44 342.53 343.39 343.96 343.18 341.88 339.65 337.81 337.69
## 1983 341.20 342.35 342.93 344.77 345.58 345.14 343.81 342.21 339.69 339.82
## 1984 343.52 344.33 345.11 346.88 347.25 346.62 345.22 343.11 340.90 341.18
## 1985 344.79 345.82 347.25 348.17 348.74 348.07 346.38 344.51 342.92 342.62
## 1986 346.11 346.78 347.68 349.37 350.03 349.37 347.76 345.73 344.68 343.99
## 1987 347.84 348.29 349.23 350.80 351.66 351.07 349.33 347.92 346.27 346.18
## 1988 350.25 351.54 352.05 353.41 354.04 353.62 352.22 350.27 348.55 348.72
## 1989 352.60 352.92 353.53 355.26 355.52 354.97 353.75 351.52 349.64 349.83
## 1990 353.50 354.55 355.23 356.04 357.00 356.07 354.67 352.76 350.82 351.04
## 1991 354.59 355.63 357.03 358.48 359.22 358.12 356.06 353.92 352.05 352.11
## 1992 355.88 356.63 357.72 359.07 359.58 359.17 356.94 354.92 352.94 353.23
## 1993 356.63 357.10 358.32 359.41 360.23 359.55 357.53 355.48 353.67 353.95
## 1994 358.34 358.89 359.95 361.25 361.67 360.94 359.55 357.49 355.84 356.00
## 1995 359.98 361.03 361.66 363.48 363.82 363.30 361.94 359.50 358.11 357.80
## 1996 362.09 363.29 364.06 364.76 365.45 365.01 363.70 361.54 359.51 359.65
## 1997 363.23 364.06 364.61 366.40 366.84 365.68 364.52 362.57 360.24 360.83
##         Nov    Dec
## 1959 314.66 315.43
## 1960 314.84 316.03
## 1961 315.94 316.85
## 1962 316.53 317.53
## 1963 316.91 318.20
## 1964 317.53 318.55
## 1965 318.70 319.25
## 1966 319.63 320.87
## 1967 320.56 321.80
## 1968 321.16 322.74
## 1969 322.69 323.95
## 1970 323.85 324.96
## 1971 324.63 325.85
## 1972 326.34 327.39
## 1973 327.99 328.48
## 1974 328.29 329.41
## 1975 329.32 330.59
## 1976 330.14 331.52
## 1977 332.24 333.68
## 1978 333.75 334.78
## 1979 335.12 336.56
## 1980 336.93 338.04
## 1981 338.19 339.44
## 1982 339.09 340.32
## 1983 340.98 342.82
## 1984 342.80 344.04
## 1985 344.06 345.38
## 1986 345.48 346.72
## 1987 347.64 348.78
## 1988 349.91 351.18
## 1989 351.14 352.37
## 1990 352.69 354.07
## 1991 353.64 354.89
## 1992 354.09 355.33
## 1993 355.30 356.78
## 1994 357.59 359.05
## 1995 359.61 360.74
## 1996 360.80 362.38
## 1997 362.49 364.34
\end{verbatim}

R will show you Mauna Loa atmospheric CO2 concentration data.

Other prebuilt objects are mathematical quantities, such as the constant
π and ∞:

\begin{Shaded}
\begin{Highlighting}[]
\NormalTok{pi}
\end{Highlighting}
\end{Shaded}

\begin{verbatim}
## [1] 3.141593
\end{verbatim}

\begin{Shaded}
\begin{Highlighting}[]
\ConstantTok{Inf}\SpecialCharTok{+}\DecValTok{1}
\end{Highlighting}
\end{Shaded}

\begin{verbatim}
## [1] Inf
\end{verbatim}

\hypertarget{variable-names}{%
\subsubsection{2.2.5 Variable names}\label{variable-names}}

We have used the letters a, b, and c as variable names, but variable
names can be almost anything. Some basic rules in R are that variable
names have to start with a letter, can't contain spaces, and should not
be variables that are predefined in R. For example, don't name one of
your variables install.packages by typing something like
install.packages \textless- 2.

A nice convention to follow is to use meaningful words that describe
what is stored, use only lower case, and use underscores as a substitute
for spaces. For the quadratic equations, we could use something like
this:

\begin{Shaded}
\begin{Highlighting}[]
\NormalTok{solution\_1 }\OtherTok{\textless{}{-}}\NormalTok{ (}\SpecialCharTok{{-}}\NormalTok{b }\SpecialCharTok{+} \FunctionTok{sqrt}\NormalTok{(b}\SpecialCharTok{\^{}}\DecValTok{2} \SpecialCharTok{{-}} \DecValTok{4}\SpecialCharTok{*}\NormalTok{a}\SpecialCharTok{*}\NormalTok{c)) }\SpecialCharTok{/}\NormalTok{ (}\DecValTok{2}\SpecialCharTok{*}\NormalTok{a)}
\NormalTok{solution\_2 }\OtherTok{\textless{}{-}}\NormalTok{ (}\SpecialCharTok{{-}}\NormalTok{b }\SpecialCharTok{{-}} \FunctionTok{sqrt}\NormalTok{(b}\SpecialCharTok{\^{}}\DecValTok{2} \SpecialCharTok{{-}} \DecValTok{4}\SpecialCharTok{*}\NormalTok{a}\SpecialCharTok{*}\NormalTok{c)) }\SpecialCharTok{/}\NormalTok{ (}\DecValTok{2}\SpecialCharTok{*}\NormalTok{a)}
\end{Highlighting}
\end{Shaded}

For more advice, we highly recommend studying Hadley Wickham's style
guide15.

\hypertarget{saving-your-workspace}{%
\subsubsection{2.2.6 Saving your
workspace}\label{saving-your-workspace}}

Values remain in the workspace until you end your session or erase them
with the function rm. But workspaces also can be saved for later use. In
fact, when you quit R, the program asks you if you want to save your
workspace. If you do save it, the next time you start R, the program
will restore the workspace.

We actually recommend against saving the workspace this way because, as
you start working on different projects, it will become harder to keep
track of what is saved. Instead, we recommend you assign the workspace a
specific name. You can do this by using the function save or save.image.
To load, use the function load. When saving a workspace, we recommend
the suffix rda or RData. In RStudio, you can also do this by navigating
to the Session tab and choosing Save Workspace as. You can later load it
using the Load Workspace options in the same tab. You can read the help
pages on save, save.image, and load to learn more.

\hypertarget{motivating-scripts}{%
\subsubsection{2.2.7 Motivating scripts}\label{motivating-scripts}}

To solve another equation such as 3x\^{}2 + 2x -1, we can copy and paste
the code above and then redefine the variables and recompute the
solution:

\begin{Shaded}
\begin{Highlighting}[]
\NormalTok{a }\OtherTok{\textless{}{-}} \DecValTok{3}
\NormalTok{b }\OtherTok{\textless{}{-}} \DecValTok{2}
\NormalTok{c }\OtherTok{\textless{}{-}} \SpecialCharTok{{-}}\DecValTok{1}
\NormalTok{(}\SpecialCharTok{{-}}\NormalTok{b }\SpecialCharTok{+} \FunctionTok{sqrt}\NormalTok{(b}\SpecialCharTok{\^{}}\DecValTok{2} \SpecialCharTok{{-}} \DecValTok{4}\SpecialCharTok{*}\NormalTok{a}\SpecialCharTok{*}\NormalTok{c)) }\SpecialCharTok{/}\NormalTok{ (}\DecValTok{2}\SpecialCharTok{*}\NormalTok{a)}
\end{Highlighting}
\end{Shaded}

\begin{verbatim}
## [1] 0.3333333
\end{verbatim}

\begin{Shaded}
\begin{Highlighting}[]
\NormalTok{(}\SpecialCharTok{{-}}\NormalTok{b }\SpecialCharTok{{-}} \FunctionTok{sqrt}\NormalTok{(b}\SpecialCharTok{\^{}}\DecValTok{2} \SpecialCharTok{{-}} \DecValTok{4}\SpecialCharTok{*}\NormalTok{a}\SpecialCharTok{*}\NormalTok{c)) }\SpecialCharTok{/}\NormalTok{ (}\DecValTok{2}\SpecialCharTok{*}\NormalTok{a)}
\end{Highlighting}
\end{Shaded}

\begin{verbatim}
## [1] -1
\end{verbatim}

By creating and saving a script with the code above, we would not need
to retype everything each time and, instead, simply change the variable
names. Try writing the script above into an editor and notice how easy
it is to change the variables and receive an answer.

\hypertarget{commenting-your-code}{%
\subsubsection{2.2.8 Commenting your code}\label{commenting-your-code}}

If a line of R code starts with the symbol \#, it is not evaluated. We
can use this to write reminders of why we wrote particular code. For
example, in the script above we could add:

\begin{Shaded}
\begin{Highlighting}[]
\DocumentationTok{\#\# Code to compute solution to quadratic equation of the form ax\^{}2 + bx + c}
\DocumentationTok{\#\# define the variables}
\NormalTok{a }\OtherTok{\textless{}{-}} \DecValTok{3} 
\NormalTok{b }\OtherTok{\textless{}{-}} \DecValTok{2}
\NormalTok{c }\OtherTok{\textless{}{-}} \SpecialCharTok{{-}}\DecValTok{1}

\DocumentationTok{\#\# now compute the solution}
\NormalTok{(}\SpecialCharTok{{-}}\NormalTok{b }\SpecialCharTok{+} \FunctionTok{sqrt}\NormalTok{(b}\SpecialCharTok{\^{}}\DecValTok{2} \SpecialCharTok{{-}} \DecValTok{4}\SpecialCharTok{*}\NormalTok{a}\SpecialCharTok{*}\NormalTok{c)) }\SpecialCharTok{/}\NormalTok{ (}\DecValTok{2}\SpecialCharTok{*}\NormalTok{a)}
\end{Highlighting}
\end{Shaded}

\begin{verbatim}
## [1] 0.3333333
\end{verbatim}

\begin{Shaded}
\begin{Highlighting}[]
\NormalTok{(}\SpecialCharTok{{-}}\NormalTok{b }\SpecialCharTok{{-}} \FunctionTok{sqrt}\NormalTok{(b}\SpecialCharTok{\^{}}\DecValTok{2} \SpecialCharTok{{-}} \DecValTok{4}\SpecialCharTok{*}\NormalTok{a}\SpecialCharTok{*}\NormalTok{c)) }\SpecialCharTok{/}\NormalTok{ (}\DecValTok{2}\SpecialCharTok{*}\NormalTok{a)}
\end{Highlighting}
\end{Shaded}

\begin{verbatim}
## [1] -1
\end{verbatim}

\hypertarget{exercises}{%
\subsection{2.3 Exercises}\label{exercises}}

\begin{enumerate}
\def\labelenumi{\arabic{enumi}.}
\tightlist
\item
  What is the sum of the first 100 positive integers? The formula for
  the sum of integers 1 through n is n(n+1)/2. Define n=100 and then use
  R to compute the sum of 1 through 100 using the formula. What is the
  sum?
\end{enumerate}

\begin{Shaded}
\begin{Highlighting}[]
\NormalTok{n}\OtherTok{=}\DecValTok{100}
\NormalTok{n}\SpecialCharTok{*}\NormalTok{(n}\SpecialCharTok{+}\DecValTok{1}\NormalTok{)}\SpecialCharTok{/}\DecValTok{2}
\end{Highlighting}
\end{Shaded}

\begin{verbatim}
## [1] 5050
\end{verbatim}

\begin{enumerate}
\def\labelenumi{\arabic{enumi}.}
\setcounter{enumi}{1}
\tightlist
\item
  Now use the same formula to compute the sum of the integers from 1
  through 1,000.
\end{enumerate}

\begin{Shaded}
\begin{Highlighting}[]
\NormalTok{n}\OtherTok{=}\DecValTok{1000}
\NormalTok{n}\SpecialCharTok{*}\NormalTok{(n}\SpecialCharTok{+}\DecValTok{1}\NormalTok{)}\SpecialCharTok{/}\DecValTok{2}
\end{Highlighting}
\end{Shaded}

\begin{verbatim}
## [1] 500500
\end{verbatim}

\begin{enumerate}
\def\labelenumi{\arabic{enumi}.}
\setcounter{enumi}{2}
\tightlist
\item
  Look at the result of typing the following code into R:
\end{enumerate}

\begin{Shaded}
\begin{Highlighting}[]
\NormalTok{n }\OtherTok{\textless{}{-}} \DecValTok{1000}
\NormalTok{x }\OtherTok{\textless{}{-}} \FunctionTok{seq}\NormalTok{(}\DecValTok{1}\NormalTok{, n)}
\FunctionTok{sum}\NormalTok{(x)}
\end{Highlighting}
\end{Shaded}

\begin{verbatim}
## [1] 500500
\end{verbatim}

Based on the result, what do you think the functions seq and sum do? You
can use help.

\begin{Shaded}
\begin{Highlighting}[]
\FunctionTok{help}\NormalTok{(seq)}
\end{Highlighting}
\end{Shaded}

\begin{Shaded}
\begin{Highlighting}[]
\FunctionTok{help}\NormalTok{(sum)}
\end{Highlighting}
\end{Shaded}

\begin{enumerate}
\def\labelenumi{\alph{enumi}.}
\tightlist
\item
  sum creates a list of numbers and seq adds them up.
\item
  seq creates a list of numbers and sum adds them up.
\item
  seq creates a random list and sum computes the sum of 1 through 1,000.
\item
  sum always returns the same number.
\end{enumerate}

answer : b

\begin{enumerate}
\def\labelenumi{\arabic{enumi}.}
\setcounter{enumi}{3}
\tightlist
\item
  In math and programming, we say that we evaluate a function when we
  replace the argument with a given number. So if we type sqrt(4), we
  evaluate the sqrt function. In R, you can evaluate a function inside
  another function. The evaluations happen from the inside out. Use one
  line of code to compute the log, in base 10, of the square root of
  100.
\end{enumerate}

\begin{Shaded}
\begin{Highlighting}[]
\FunctionTok{log}\NormalTok{(}\FunctionTok{sqrt}\NormalTok{(}\DecValTok{100}\NormalTok{),}\DecValTok{10}\NormalTok{)}
\end{Highlighting}
\end{Shaded}

\begin{verbatim}
## [1] 1
\end{verbatim}

\begin{enumerate}
\def\labelenumi{\arabic{enumi}.}
\setcounter{enumi}{4}
\tightlist
\item
  Which of the following will always return the numeric value stored in
  x? You can try out examples and use the help system if you want.
\end{enumerate}

\begin{enumerate}
\def\labelenumi{\alph{enumi}.}
\tightlist
\item
  log(10\^{}x)
\item
  log10(x\^{}10)
\item
  log(exp(x))
\item
  exp(log(x, base = 2))
\end{enumerate}

\begin{Shaded}
\begin{Highlighting}[]
\NormalTok{x}\OtherTok{=}\DecValTok{1}
\FunctionTok{log}\NormalTok{(}\DecValTok{10}\SpecialCharTok{\^{}}\NormalTok{x)}
\end{Highlighting}
\end{Shaded}

\begin{verbatim}
## [1] 2.302585
\end{verbatim}

\begin{Shaded}
\begin{Highlighting}[]
\FunctionTok{log10}\NormalTok{(x}\SpecialCharTok{\^{}}\DecValTok{10}\NormalTok{)}
\end{Highlighting}
\end{Shaded}

\begin{verbatim}
## [1] 0
\end{verbatim}

\begin{Shaded}
\begin{Highlighting}[]
\FunctionTok{log}\NormalTok{(}\FunctionTok{exp}\NormalTok{(x))}
\end{Highlighting}
\end{Shaded}

\begin{verbatim}
## [1] 1
\end{verbatim}

\begin{Shaded}
\begin{Highlighting}[]
\FunctionTok{exp}\NormalTok{(}\FunctionTok{log}\NormalTok{(x, }\AttributeTok{base =} \DecValTok{2}\NormalTok{))}
\end{Highlighting}
\end{Shaded}

\begin{verbatim}
## [1] 1
\end{verbatim}

\begin{Shaded}
\begin{Highlighting}[]
\NormalTok{x}\OtherTok{=}\DecValTok{2}
\FunctionTok{log}\NormalTok{(}\DecValTok{10}\SpecialCharTok{\^{}}\NormalTok{x)}
\end{Highlighting}
\end{Shaded}

\begin{verbatim}
## [1] 4.60517
\end{verbatim}

\begin{Shaded}
\begin{Highlighting}[]
\FunctionTok{log10}\NormalTok{(x}\SpecialCharTok{\^{}}\DecValTok{10}\NormalTok{)}
\end{Highlighting}
\end{Shaded}

\begin{verbatim}
## [1] 3.0103
\end{verbatim}

\begin{Shaded}
\begin{Highlighting}[]
\FunctionTok{log}\NormalTok{(}\FunctionTok{exp}\NormalTok{(x))}
\end{Highlighting}
\end{Shaded}

\begin{verbatim}
## [1] 2
\end{verbatim}

\begin{Shaded}
\begin{Highlighting}[]
\FunctionTok{exp}\NormalTok{(}\FunctionTok{log}\NormalTok{(x, }\AttributeTok{base =} \DecValTok{2}\NormalTok{))}
\end{Highlighting}
\end{Shaded}

\begin{verbatim}
## [1] 2.718282
\end{verbatim}

\begin{Shaded}
\begin{Highlighting}[]
\NormalTok{x}\OtherTok{=}\DecValTok{3}
\FunctionTok{log}\NormalTok{(}\DecValTok{10}\SpecialCharTok{\^{}}\NormalTok{x)}
\end{Highlighting}
\end{Shaded}

\begin{verbatim}
## [1] 6.907755
\end{verbatim}

\begin{Shaded}
\begin{Highlighting}[]
\FunctionTok{log10}\NormalTok{(x}\SpecialCharTok{\^{}}\DecValTok{10}\NormalTok{)}
\end{Highlighting}
\end{Shaded}

\begin{verbatim}
## [1] 4.771213
\end{verbatim}

\begin{Shaded}
\begin{Highlighting}[]
\FunctionTok{log}\NormalTok{(}\FunctionTok{exp}\NormalTok{(x))}
\end{Highlighting}
\end{Shaded}

\begin{verbatim}
## [1] 3
\end{verbatim}

\begin{Shaded}
\begin{Highlighting}[]
\FunctionTok{exp}\NormalTok{(}\FunctionTok{log}\NormalTok{(x, }\AttributeTok{base =} \DecValTok{2}\NormalTok{))}
\end{Highlighting}
\end{Shaded}

\begin{verbatim}
## [1] 4.879108
\end{verbatim}

answer : c

\hypertarget{data-types}{%
\subsection{2.4 Data types}\label{data-types}}

Variables in R can be of different types. For example, we need to
distinguish numbers from character strings and tables from simple lists
of numbers. The function class helps us determine what type of object we
have:

\begin{Shaded}
\begin{Highlighting}[]
\NormalTok{a }\OtherTok{\textless{}{-}} \DecValTok{2}
\FunctionTok{class}\NormalTok{(a)}
\end{Highlighting}
\end{Shaded}

\begin{verbatim}
## [1] "numeric"
\end{verbatim}

To work efficiently in R, it is important to learn the different types
of variables and what we can do with these.

\hypertarget{data-frames}{%
\subsubsection{2.4.1 Data frames}\label{data-frames}}

Up to now, the variables we have defined are just one number. This is
not very useful for storing data. The most common way of storing a
dataset in R is in a data frame. Conceptually, we can think of a data
frame as a table with rows representing observations and the different
variables reported for each observation defining the columns. Data
frames are particularly useful for datasets because we can combine
different data types into one object.

A large proportion of data analysis challenges start with data stored in
a data frame. For example, we stored the data for our motivating example
in a data frame. You can access this dataset by loading the dslabs
library and loading the murders dataset using the data function:

\begin{Shaded}
\begin{Highlighting}[]
\FunctionTok{library}\NormalTok{(dslabs)}
\end{Highlighting}
\end{Shaded}

\begin{verbatim}
## Warning: 패키지 'dslabs'는 R 버전 4.1.1에서 작성되었습니다
\end{verbatim}

\begin{Shaded}
\begin{Highlighting}[]
\FunctionTok{data}\NormalTok{(murders)}
\end{Highlighting}
\end{Shaded}

To see that this is in fact a data frame, we type:

\begin{Shaded}
\begin{Highlighting}[]
\FunctionTok{class}\NormalTok{(murders)}
\end{Highlighting}
\end{Shaded}

\begin{verbatim}
## [1] "data.frame"
\end{verbatim}

\hypertarget{examining-an-object}{%
\subsubsection{2.4.2 Examining an object}\label{examining-an-object}}

The function str is useful for finding out more about the structure of
an object:

\begin{Shaded}
\begin{Highlighting}[]
\FunctionTok{str}\NormalTok{(murders)}
\end{Highlighting}
\end{Shaded}

\begin{verbatim}
## 'data.frame':    51 obs. of  5 variables:
##  $ state     : chr  "Alabama" "Alaska" "Arizona" "Arkansas" ...
##  $ abb       : chr  "AL" "AK" "AZ" "AR" ...
##  $ region    : Factor w/ 4 levels "Northeast","South",..: 2 4 4 2 4 4 1 2 2 2 ...
##  $ population: num  4779736 710231 6392017 2915918 37253956 ...
##  $ total     : num  135 19 232 93 1257 ...
\end{verbatim}

This tells us much more about the object. We see that the table has 51
rows (50 states plus DC) and five variables. We can show the first six
lines using the function head:

\begin{Shaded}
\begin{Highlighting}[]
\FunctionTok{head}\NormalTok{(murders)}
\end{Highlighting}
\end{Shaded}

\begin{verbatim}
##        state abb region population total
## 1    Alabama  AL  South    4779736   135
## 2     Alaska  AK   West     710231    19
## 3    Arizona  AZ   West    6392017   232
## 4   Arkansas  AR  South    2915918    93
## 5 California  CA   West   37253956  1257
## 6   Colorado  CO   West    5029196    65
\end{verbatim}

In this dataset, each state is considered an observation and five
variables are reported for each state.

Before we go any further in answering our original question about
different states, let's learn more about the components of this object.

\hypertarget{the-accessor}{%
\subsubsection{2.4.3 The accessor: \$}\label{the-accessor}}

For our analysis, we will need to access the different variables
represented by columns included in this data frame. To do this, we use
the accessor operator \$ in the following way:

\begin{Shaded}
\begin{Highlighting}[]
\NormalTok{murders}\SpecialCharTok{$}\NormalTok{population}
\end{Highlighting}
\end{Shaded}

\begin{verbatim}
##  [1]  4779736   710231  6392017  2915918 37253956  5029196  3574097   897934
##  [9]   601723 19687653  9920000  1360301  1567582 12830632  6483802  3046355
## [17]  2853118  4339367  4533372  1328361  5773552  6547629  9883640  5303925
## [25]  2967297  5988927   989415  1826341  2700551  1316470  8791894  2059179
## [33] 19378102  9535483   672591 11536504  3751351  3831074 12702379  1052567
## [41]  4625364   814180  6346105 25145561  2763885   625741  8001024  6724540
## [49]  1852994  5686986   563626
\end{verbatim}

But how did we know to use population? Previously, by applying the
function str to the object murders, we revealed the names for each of
the five variables stored in this table. We can quickly access the
variable names using:

\begin{Shaded}
\begin{Highlighting}[]
\FunctionTok{names}\NormalTok{(murders)}
\end{Highlighting}
\end{Shaded}

\begin{verbatim}
## [1] "state"      "abb"        "region"     "population" "total"
\end{verbatim}

It is important to know that the order of the entries in
murders\$population preserves the order of the rows in our data table.
This will later permit us to manipulate one variable based on the
results of another. For example, we will be able to order the state
names by the number of murders.

Tip: R comes with a very nice auto-complete functionality that saves us
the trouble of typing out all the names. Try typing murders\$p then
hitting the tab key on your keyboard. This functionality and many other
useful auto-complete features are available when working in RStudio.

\hypertarget{vectors-numerics-characters-and-logical}{%
\subsubsection{2.4.4 Vectors: numerics, characters, and
logical}\label{vectors-numerics-characters-and-logical}}

The object murders\$population is not one number but several. We call
these types of objects vectors. A single number is technically a vector
of length 1, but in general we use the term vectors to refer to objects
with several entries. The function length tells you how many entries are
in the vector:

\begin{Shaded}
\begin{Highlighting}[]
\NormalTok{pop }\OtherTok{\textless{}{-}}\NormalTok{ murders}\SpecialCharTok{$}\NormalTok{population}
\FunctionTok{length}\NormalTok{(pop)}
\end{Highlighting}
\end{Shaded}

\begin{verbatim}
## [1] 51
\end{verbatim}

This particular vector is numeric since population sizes are numbers:

\begin{Shaded}
\begin{Highlighting}[]
\FunctionTok{class}\NormalTok{(pop)}
\end{Highlighting}
\end{Shaded}

\begin{verbatim}
## [1] "numeric"
\end{verbatim}

In a numeric vector, every entry must be a number.

To store character strings, vectors can also be of class character. For
example, the state names are characters:

\begin{Shaded}
\begin{Highlighting}[]
\FunctionTok{class}\NormalTok{(murders}\SpecialCharTok{$}\NormalTok{state)}
\end{Highlighting}
\end{Shaded}

\begin{verbatim}
## [1] "character"
\end{verbatim}

As with numeric vectors, all entries in a character vector need to be a
character.

Another important type of vectors are logical vectors. These must be
either TRUE or FALSE.

\begin{Shaded}
\begin{Highlighting}[]
\NormalTok{z }\OtherTok{\textless{}{-}} \DecValTok{3} \SpecialCharTok{==} \DecValTok{2}
\NormalTok{z}
\end{Highlighting}
\end{Shaded}

\begin{verbatim}
## [1] FALSE
\end{verbatim}

\begin{Shaded}
\begin{Highlighting}[]
\FunctionTok{class}\NormalTok{(z)}
\end{Highlighting}
\end{Shaded}

\begin{verbatim}
## [1] "logical"
\end{verbatim}

Here the == is a relational operator asking if 3 is equal to 2. In R, if
you just use one =, you actually assign a variable, but if you use two
== you test for equality.

You can see the other relational operators by typing:

\begin{Shaded}
\begin{Highlighting}[]
\NormalTok{?Comparison}
\end{Highlighting}
\end{Shaded}

In future sections, you will see how useful relational operators can be.

We discuss more important features of vectors after the next set of
exercises.

Advanced: Mathematically, the values in pop are integers and there is an
integer class in R. However, by default, numbers are assigned class
numeric even when they are round integers. For example, class(1) returns
numeric. You can turn them into class integer with the as.integer()
function or by adding an L like this: 1L. Note the class by typing:
class(1L)

\begin{Shaded}
\begin{Highlighting}[]
\FunctionTok{class}\NormalTok{(1L)}
\end{Highlighting}
\end{Shaded}

\begin{verbatim}
## [1] "integer"
\end{verbatim}

\hypertarget{factors}{%
\subsubsection{2.4.5 Factors}\label{factors}}

In the murders dataset, we might expect the region to also be a
character vector. However, it is not:

\begin{Shaded}
\begin{Highlighting}[]
\FunctionTok{class}\NormalTok{(murders}\SpecialCharTok{$}\NormalTok{region)}
\end{Highlighting}
\end{Shaded}

\begin{verbatim}
## [1] "factor"
\end{verbatim}

It is a factor. Factors are useful for storing categorical data. We can
see that there are only 4 regions by using the levels function:

\begin{Shaded}
\begin{Highlighting}[]
\FunctionTok{levels}\NormalTok{(murders}\SpecialCharTok{$}\NormalTok{region)}
\end{Highlighting}
\end{Shaded}

\begin{verbatim}
## [1] "Northeast"     "South"         "North Central" "West"
\end{verbatim}

In the background, R stores these levels as integers and keeps a map to
keep track of the labels. This is more memory efficient than storing all
the characters.

Note that the levels have an order that is different from the order of
appearance in the factor object. The default in R is for the levels to
follow alphabetical order. However, often we want the levels to follow a
different order. You can specify an order through the levels argument
when creating the factor with the factor function. For example, in the
murders dataset regions are ordered from east to west. The function
reorder lets us change the order of the levels of a factor variable
based on a summary computed on a numeric vector. We will demonstrate
this with a simple example, and will see more advanced ones in the Data
Visualization part of the book.

Suppose we want the levels of the region by the total number of murders
rather than alphabetical order. If there are values associated with each
level, we can use the reorder and specify a data summary to determine
the order. The following code takes the sum of the total murders in each
region, and reorders the factor following these sums.

\begin{Shaded}
\begin{Highlighting}[]
\NormalTok{region }\OtherTok{\textless{}{-}}\NormalTok{ murders}\SpecialCharTok{$}\NormalTok{region}
\NormalTok{value }\OtherTok{\textless{}{-}}\NormalTok{ murders}\SpecialCharTok{$}\NormalTok{total}
\NormalTok{region }\OtherTok{\textless{}{-}} \FunctionTok{reorder}\NormalTok{(region, value, }\AttributeTok{FUN =}\NormalTok{ sum)}
\FunctionTok{levels}\NormalTok{(region)}
\end{Highlighting}
\end{Shaded}

\begin{verbatim}
## [1] "Northeast"     "North Central" "West"          "South"
\end{verbatim}

The new order is in agreement with the fact that the Northeast has the
least murders and the South has the most.

Warning: Factors can be a source of confusion since sometimes they
behave like characters and sometimes they do not. As a result, confusing
factors and characters are a common source of bugs.

\hypertarget{lists}{%
\subsubsection{2.4.6 Lists}\label{lists}}

Data frames are a special case of lists. Lists are useful because you
can store any combination of different types. You can create a list
using the list function like this:

\begin{Shaded}
\begin{Highlighting}[]
\NormalTok{record }\OtherTok{\textless{}{-}} \FunctionTok{list}\NormalTok{(}\AttributeTok{name =} \StringTok{"John Doe"}\NormalTok{,}
             \AttributeTok{student\_id =} \DecValTok{1234}\NormalTok{,}
             \AttributeTok{grades =} \FunctionTok{c}\NormalTok{(}\DecValTok{95}\NormalTok{, }\DecValTok{82}\NormalTok{, }\DecValTok{91}\NormalTok{, }\DecValTok{97}\NormalTok{, }\DecValTok{93}\NormalTok{),}
             \AttributeTok{final\_grade =} \StringTok{"A"}\NormalTok{)}
\end{Highlighting}
\end{Shaded}

The function c is described in Section 2.6.

This list includes a character, a number, a vector with five numbers,
and another character.

\begin{Shaded}
\begin{Highlighting}[]
\NormalTok{record}
\end{Highlighting}
\end{Shaded}

\begin{verbatim}
## $name
## [1] "John Doe"
## 
## $student_id
## [1] 1234
## 
## $grades
## [1] 95 82 91 97 93
## 
## $final_grade
## [1] "A"
\end{verbatim}

As with data frames, you can extract the components of a list with the
accessor \$.

\begin{Shaded}
\begin{Highlighting}[]
\NormalTok{record}\SpecialCharTok{$}\NormalTok{student\_id}
\end{Highlighting}
\end{Shaded}

\begin{verbatim}
## [1] 1234
\end{verbatim}

We can also use double square brackets ({[}{[}) like this:

\begin{Shaded}
\begin{Highlighting}[]
\NormalTok{record[[}\StringTok{"student\_id"}\NormalTok{]]}
\end{Highlighting}
\end{Shaded}

\begin{verbatim}
## [1] 1234
\end{verbatim}

You should get used to the fact that in R, there are often several ways
to do the same thing, such as accessing entries.

You might also encounter lists without variable names.

\begin{Shaded}
\begin{Highlighting}[]
\NormalTok{record2 }\OtherTok{\textless{}{-}} \FunctionTok{list}\NormalTok{(}\StringTok{"John Doe"}\NormalTok{, }\DecValTok{1234}\NormalTok{)}
\NormalTok{record2}
\end{Highlighting}
\end{Shaded}

\begin{verbatim}
## [[1]]
## [1] "John Doe"
## 
## [[2]]
## [1] 1234
\end{verbatim}

If a list does not have names, you cannot extract the elements with \$,
but you can still use the brackets method and instead of providing the
variable name, you provide the list index, like this:

\begin{Shaded}
\begin{Highlighting}[]
\NormalTok{record2[[}\DecValTok{1}\NormalTok{]]}
\end{Highlighting}
\end{Shaded}

\begin{verbatim}
## [1] "John Doe"
\end{verbatim}

We won't be using lists until later, but you might encounter one in your
own exploration of R. For this reason, we show you some basics here.

\hypertarget{matrices}{%
\subsubsection{2.4.7 Matrices}\label{matrices}}

Matrices are another type of object that are common in R. Matrices are
similar to data frames in that they are two-dimensional: they have rows
and columns. However, like numeric, character and logical vectors,
entries in matrices have to be all the same type. For this reason data
frames are much more useful for storing data, since we can have
characters, factors, and numbers in them.

Yet matrices have a major advantage over data frames: we can perform
matrix algebra operations, a powerful type of mathematical technique. We
do not describe these operations in this book, but much of what happens
in the background when you perform a data analysis involves matrices. We
cover matrices in more detail in Chapter 33.1 but describe them briefly
here since some of the functions we will learn return matrices.

We can define a matrix using the matrix function. We need to specify the
number of rows and columns.

\begin{Shaded}
\begin{Highlighting}[]
\NormalTok{mat }\OtherTok{\textless{}{-}} \FunctionTok{matrix}\NormalTok{(}\DecValTok{1}\SpecialCharTok{:}\DecValTok{12}\NormalTok{, }\DecValTok{4}\NormalTok{, }\DecValTok{3}\NormalTok{)}
\NormalTok{mat}
\end{Highlighting}
\end{Shaded}

\begin{verbatim}
##      [,1] [,2] [,3]
## [1,]    1    5    9
## [2,]    2    6   10
## [3,]    3    7   11
## [4,]    4    8   12
\end{verbatim}

You can access specific entries in a matrix using square brackets ({[}).
If you want the second row, third column, you use:

\begin{Shaded}
\begin{Highlighting}[]
\NormalTok{mat[}\DecValTok{2}\NormalTok{, }\DecValTok{3}\NormalTok{]}
\end{Highlighting}
\end{Shaded}

\begin{verbatim}
## [1] 10
\end{verbatim}

If you want the entire second row, you leave the column spot empty:

\begin{Shaded}
\begin{Highlighting}[]
\NormalTok{mat[}\DecValTok{2}\NormalTok{, ]}
\end{Highlighting}
\end{Shaded}

\begin{verbatim}
## [1]  2  6 10
\end{verbatim}

Notice that this returns a vector, not a matrix.

Similarly, if you want the entire third column, you leave the row spot
empty:

\begin{Shaded}
\begin{Highlighting}[]
\NormalTok{mat[, }\DecValTok{3}\NormalTok{]}
\end{Highlighting}
\end{Shaded}

\begin{verbatim}
## [1]  9 10 11 12
\end{verbatim}

This is also a vector, not a matrix.

You can access more than one column or more than one row if you like.
This will give you a new matrix.

\begin{Shaded}
\begin{Highlighting}[]
\NormalTok{mat[, }\DecValTok{2}\SpecialCharTok{:}\DecValTok{3}\NormalTok{]}
\end{Highlighting}
\end{Shaded}

\begin{verbatim}
##      [,1] [,2]
## [1,]    5    9
## [2,]    6   10
## [3,]    7   11
## [4,]    8   12
\end{verbatim}

You can subset both rows and columns:

\begin{Shaded}
\begin{Highlighting}[]
\NormalTok{mat[}\DecValTok{1}\SpecialCharTok{:}\DecValTok{2}\NormalTok{, }\DecValTok{2}\SpecialCharTok{:}\DecValTok{3}\NormalTok{]}
\end{Highlighting}
\end{Shaded}

\begin{verbatim}
##      [,1] [,2]
## [1,]    5    9
## [2,]    6   10
\end{verbatim}

We can convert matrices into data frames using the function
as.data.frame:

\begin{Shaded}
\begin{Highlighting}[]
\FunctionTok{as.data.frame}\NormalTok{(mat)}
\end{Highlighting}
\end{Shaded}

\begin{verbatim}
##   V1 V2 V3
## 1  1  5  9
## 2  2  6 10
## 3  3  7 11
## 4  4  8 12
\end{verbatim}

You can also use single square brackets ({[}) to access rows and columns
of a data frame:

\begin{Shaded}
\begin{Highlighting}[]
\FunctionTok{data}\NormalTok{(}\StringTok{"murders"}\NormalTok{)}
\NormalTok{murders[}\DecValTok{25}\NormalTok{, }\DecValTok{1}\NormalTok{]}
\end{Highlighting}
\end{Shaded}

\begin{verbatim}
## [1] "Mississippi"
\end{verbatim}

\begin{Shaded}
\begin{Highlighting}[]
\NormalTok{murders[}\DecValTok{2}\SpecialCharTok{:}\DecValTok{3}\NormalTok{, ]}
\end{Highlighting}
\end{Shaded}

\begin{verbatim}
##     state abb region population total
## 2  Alaska  AK   West     710231    19
## 3 Arizona  AZ   West    6392017   232
\end{verbatim}

\hypertarget{exercises-1}{%
\subsection{2.5 Exercises}\label{exercises-1}}

\begin{enumerate}
\def\labelenumi{\arabic{enumi}.}
\tightlist
\item
  Load the US murders dataset.
\end{enumerate}

\begin{Shaded}
\begin{Highlighting}[]
\FunctionTok{library}\NormalTok{(dslabs)}
\FunctionTok{data}\NormalTok{(murders)}
\end{Highlighting}
\end{Shaded}

Use the function str to examine the structure of the murders object.
Which of the following best describes the variables represented in this
data frame?

\begin{Shaded}
\begin{Highlighting}[]
\FunctionTok{str}\NormalTok{(murders)}
\end{Highlighting}
\end{Shaded}

\begin{verbatim}
## 'data.frame':    51 obs. of  5 variables:
##  $ state     : chr  "Alabama" "Alaska" "Arizona" "Arkansas" ...
##  $ abb       : chr  "AL" "AK" "AZ" "AR" ...
##  $ region    : Factor w/ 4 levels "Northeast","South",..: 2 4 4 2 4 4 1 2 2 2 ...
##  $ population: num  4779736 710231 6392017 2915918 37253956 ...
##  $ total     : num  135 19 232 93 1257 ...
\end{verbatim}

\begin{Shaded}
\begin{Highlighting}[]
\FunctionTok{head}\NormalTok{(murders)}
\end{Highlighting}
\end{Shaded}

\begin{verbatim}
##        state abb region population total
## 1    Alabama  AL  South    4779736   135
## 2     Alaska  AK   West     710231    19
## 3    Arizona  AZ   West    6392017   232
## 4   Arkansas  AR  South    2915918    93
## 5 California  CA   West   37253956  1257
## 6   Colorado  CO   West    5029196    65
\end{verbatim}

\begin{enumerate}
\def\labelenumi{\alph{enumi}.}
\tightlist
\item
  The 51 states.
\item
  The murder rates for all 50 states and DC.
\item
  The state name, the abbreviation of the state name, the state's
  region, and the state's population and total number of murders for
  2010.
\item
  str shows no relevant information.
\end{enumerate}

answer : c

\begin{enumerate}
\def\labelenumi{\arabic{enumi}.}
\setcounter{enumi}{1}
\tightlist
\item
  What are the column names used by the data frame for these five
  variables?
\end{enumerate}

\begin{Shaded}
\begin{Highlighting}[]
\FunctionTok{colnames}\NormalTok{(murders)}
\end{Highlighting}
\end{Shaded}

\begin{verbatim}
## [1] "state"      "abb"        "region"     "population" "total"
\end{verbatim}

\begin{enumerate}
\def\labelenumi{\arabic{enumi}.}
\setcounter{enumi}{2}
\tightlist
\item
  Use the accessor \$ to extract the state abbreviations and assign them
  to the object a. What is the class of this object?
\end{enumerate}

\begin{Shaded}
\begin{Highlighting}[]
\NormalTok{a }\OtherTok{\textless{}{-}}\NormalTok{ murders}\SpecialCharTok{$}\NormalTok{abb}
\FunctionTok{class}\NormalTok{(a)}
\end{Highlighting}
\end{Shaded}

\begin{verbatim}
## [1] "character"
\end{verbatim}

\begin{enumerate}
\def\labelenumi{\arabic{enumi}.}
\setcounter{enumi}{3}
\tightlist
\item
  Now use the square brackets to extract the state abbreviations and
  assign them to the object b. Use the identical function to determine
  if a and b are the same.
\end{enumerate}

\begin{Shaded}
\begin{Highlighting}[]
\NormalTok{b }\OtherTok{\textless{}{-}}\NormalTok{ murders[[}\StringTok{"abb"}\NormalTok{]]}
\FunctionTok{identical}\NormalTok{(a,b)}
\end{Highlighting}
\end{Shaded}

\begin{verbatim}
## [1] TRUE
\end{verbatim}

\begin{enumerate}
\def\labelenumi{\arabic{enumi}.}
\setcounter{enumi}{4}
\tightlist
\item
  We saw that the region column stores a factor. You can corroborate
  this by typing:
\end{enumerate}

\begin{Shaded}
\begin{Highlighting}[]
\FunctionTok{class}\NormalTok{(murders}\SpecialCharTok{$}\NormalTok{region)}
\end{Highlighting}
\end{Shaded}

\begin{verbatim}
## [1] "factor"
\end{verbatim}

With one line of code, use the function levels and length to determine
the number of regions defined by this dataset.

\begin{Shaded}
\begin{Highlighting}[]
\FunctionTok{length}\NormalTok{(}\FunctionTok{levels}\NormalTok{(murders}\SpecialCharTok{$}\NormalTok{region))}
\end{Highlighting}
\end{Shaded}

\begin{verbatim}
## [1] 4
\end{verbatim}

\begin{enumerate}
\def\labelenumi{\arabic{enumi}.}
\setcounter{enumi}{5}
\tightlist
\item
  The function table takes a vector and returns the frequency of each
  element. You can quickly see how many states are in each region by
  applying this function. Use this function in one line of code to
  create a table of states per region.
\end{enumerate}

\begin{Shaded}
\begin{Highlighting}[]
\FunctionTok{table}\NormalTok{(murders}\SpecialCharTok{$}\NormalTok{region)}
\end{Highlighting}
\end{Shaded}

\begin{verbatim}
## 
##     Northeast         South North Central          West 
##             9            17            12            13
\end{verbatim}

\hypertarget{vectors}{%
\subsection{2.6 Vectors}\label{vectors}}

In R, the most basic objects available to store data are vectors. As we
have seen, complex datasets can usually be broken down into components
that are vectors. For example, in a data frame, each column is a vector.
Here we learn more about this important class.

\hypertarget{creating-vectors}{%
\subsubsection{2.6.1 Creating vectors}\label{creating-vectors}}

We can create vectors using the function c, which stands for
concatenate. We use c to concatenate entries in the following way:

\begin{Shaded}
\begin{Highlighting}[]
\NormalTok{codes }\OtherTok{\textless{}{-}} \FunctionTok{c}\NormalTok{(}\DecValTok{380}\NormalTok{, }\DecValTok{124}\NormalTok{, }\DecValTok{818}\NormalTok{)}
\NormalTok{codes}
\end{Highlighting}
\end{Shaded}

\begin{verbatim}
## [1] 380 124 818
\end{verbatim}

We can also create character vectors. We use the quotes to denote that
the entries are characters rather than variable names.

\begin{Shaded}
\begin{Highlighting}[]
\NormalTok{country }\OtherTok{\textless{}{-}} \FunctionTok{c}\NormalTok{(}\StringTok{"italy"}\NormalTok{, }\StringTok{"canada"}\NormalTok{, }\StringTok{"egypt"}\NormalTok{)}
\end{Highlighting}
\end{Shaded}

In R you can also use single quotes:

\begin{Shaded}
\begin{Highlighting}[]
\NormalTok{country }\OtherTok{\textless{}{-}} \FunctionTok{c}\NormalTok{(}\StringTok{\textquotesingle{}italy\textquotesingle{}}\NormalTok{, }\StringTok{\textquotesingle{}canada\textquotesingle{}}\NormalTok{, }\StringTok{\textquotesingle{}egypt\textquotesingle{}}\NormalTok{)}
\end{Highlighting}
\end{Shaded}

But be careful not to confuse the single quote ' with the back quote `.

By now you should know that if you type:

\begin{Shaded}
\begin{Highlighting}[]
\CommentTok{\# country \textless{}{-} c(italy, canada, egypt)}
\end{Highlighting}
\end{Shaded}

you receive an error because the variables italy, canada, and egypt are
not defined. If we do not use the quotes, R looks for variables with
those names and returns an error.

\hypertarget{names}{%
\subsubsection{2.6.2 Names}\label{names}}

Sometimes it is useful to name the entries of a vector. For example,
when defining a vector of country codes, we can use the names to connect
the two:

\begin{Shaded}
\begin{Highlighting}[]
\NormalTok{codes }\OtherTok{\textless{}{-}} \FunctionTok{c}\NormalTok{(}\AttributeTok{italy =} \DecValTok{380}\NormalTok{, }\AttributeTok{canada =} \DecValTok{124}\NormalTok{, }\AttributeTok{egypt =} \DecValTok{818}\NormalTok{)}
\NormalTok{codes}
\end{Highlighting}
\end{Shaded}

\begin{verbatim}
##  italy canada  egypt 
##    380    124    818
\end{verbatim}

The object codes continues to be a numeric vector:

\begin{Shaded}
\begin{Highlighting}[]
\FunctionTok{class}\NormalTok{(codes)}
\end{Highlighting}
\end{Shaded}

\begin{verbatim}
## [1] "numeric"
\end{verbatim}

but with names:

\begin{Shaded}
\begin{Highlighting}[]
\FunctionTok{names}\NormalTok{(codes)}
\end{Highlighting}
\end{Shaded}

\begin{verbatim}
## [1] "italy"  "canada" "egypt"
\end{verbatim}

If the use of strings without quotes looks confusing, know that you can
use the quotes as well:

\begin{Shaded}
\begin{Highlighting}[]
\NormalTok{codes }\OtherTok{\textless{}{-}} \FunctionTok{c}\NormalTok{(}\StringTok{"italy"} \OtherTok{=} \DecValTok{380}\NormalTok{, }\StringTok{"canada"} \OtherTok{=} \DecValTok{124}\NormalTok{, }\StringTok{"egypt"} \OtherTok{=} \DecValTok{818}\NormalTok{)}
\NormalTok{codes}
\end{Highlighting}
\end{Shaded}

\begin{verbatim}
##  italy canada  egypt 
##    380    124    818
\end{verbatim}

There is no difference between this function call and the previous one.
This is one of the many ways in which R is quirky compared to other
languages.

We can also assign names using the names functions:

\begin{Shaded}
\begin{Highlighting}[]
\NormalTok{codes }\OtherTok{\textless{}{-}} \FunctionTok{c}\NormalTok{(}\DecValTok{380}\NormalTok{, }\DecValTok{124}\NormalTok{, }\DecValTok{818}\NormalTok{)}
\NormalTok{country }\OtherTok{\textless{}{-}} \FunctionTok{c}\NormalTok{(}\StringTok{"italy"}\NormalTok{,}\StringTok{"canada"}\NormalTok{,}\StringTok{"egypt"}\NormalTok{)}
\FunctionTok{names}\NormalTok{(codes) }\OtherTok{\textless{}{-}}\NormalTok{ country}
\NormalTok{codes}
\end{Highlighting}
\end{Shaded}

\begin{verbatim}
##  italy canada  egypt 
##    380    124    818
\end{verbatim}

\hypertarget{sequences}{%
\subsubsection{2.6.3 Sequences}\label{sequences}}

Another useful function for creating vectors generates sequences:

\begin{Shaded}
\begin{Highlighting}[]
\FunctionTok{seq}\NormalTok{(}\DecValTok{1}\NormalTok{, }\DecValTok{10}\NormalTok{)}
\end{Highlighting}
\end{Shaded}

\begin{verbatim}
##  [1]  1  2  3  4  5  6  7  8  9 10
\end{verbatim}

The first argument defines the start, and the second defines the end
which is included. The default is to go up in increments of 1, but a
third argument lets us tell it how much to jump by:

\begin{Shaded}
\begin{Highlighting}[]
\FunctionTok{seq}\NormalTok{(}\DecValTok{1}\NormalTok{, }\DecValTok{10}\NormalTok{, }\DecValTok{2}\NormalTok{)}
\end{Highlighting}
\end{Shaded}

\begin{verbatim}
## [1] 1 3 5 7 9
\end{verbatim}

If we want consecutive integers, we can use the following shorthand:

\begin{Shaded}
\begin{Highlighting}[]
\DecValTok{1}\SpecialCharTok{:}\DecValTok{10}
\end{Highlighting}
\end{Shaded}

\begin{verbatim}
##  [1]  1  2  3  4  5  6  7  8  9 10
\end{verbatim}

When we use these functions, R produces integers, not numerics, because
they are typically used to index something:

\begin{Shaded}
\begin{Highlighting}[]
\FunctionTok{class}\NormalTok{(}\DecValTok{1}\SpecialCharTok{:}\DecValTok{10}\NormalTok{)}
\end{Highlighting}
\end{Shaded}

\begin{verbatim}
## [1] "integer"
\end{verbatim}

However, if we create a sequence including non-integers, the class
changes:

\begin{Shaded}
\begin{Highlighting}[]
\FunctionTok{class}\NormalTok{(}\FunctionTok{seq}\NormalTok{(}\DecValTok{1}\NormalTok{, }\DecValTok{10}\NormalTok{, }\FloatTok{0.5}\NormalTok{))}
\end{Highlighting}
\end{Shaded}

\begin{verbatim}
## [1] "numeric"
\end{verbatim}

\hypertarget{subsetting}{%
\subsubsection{2.6.4 Subsetting}\label{subsetting}}

We use square brackets to access specific elements of a vector. For the
vector codes we defined above, we can access the second element using:

\begin{Shaded}
\begin{Highlighting}[]
\NormalTok{codes[}\DecValTok{2}\NormalTok{]}
\end{Highlighting}
\end{Shaded}

\begin{verbatim}
## canada 
##    124
\end{verbatim}

You can get more than one entry by using a multi-entry vector as an
index:

\begin{Shaded}
\begin{Highlighting}[]
\NormalTok{codes[}\FunctionTok{c}\NormalTok{(}\DecValTok{1}\NormalTok{,}\DecValTok{3}\NormalTok{)]}
\end{Highlighting}
\end{Shaded}

\begin{verbatim}
## italy egypt 
##   380   818
\end{verbatim}

The sequences defined above are particularly useful if we want to
access, say, the first two elements:

\begin{Shaded}
\begin{Highlighting}[]
\NormalTok{codes[}\DecValTok{1}\SpecialCharTok{:}\DecValTok{2}\NormalTok{]}
\end{Highlighting}
\end{Shaded}

\begin{verbatim}
##  italy canada 
##    380    124
\end{verbatim}

If the elements have names, we can also access the entries using these
names. Below are two examples.

\begin{Shaded}
\begin{Highlighting}[]
\NormalTok{codes[}\StringTok{"canada"}\NormalTok{]}
\end{Highlighting}
\end{Shaded}

\begin{verbatim}
## canada 
##    124
\end{verbatim}

\begin{Shaded}
\begin{Highlighting}[]
\NormalTok{codes[}\FunctionTok{c}\NormalTok{(}\StringTok{"egypt"}\NormalTok{,}\StringTok{"italy"}\NormalTok{)]}
\end{Highlighting}
\end{Shaded}

\begin{verbatim}
## egypt italy 
##   818   380
\end{verbatim}

\hypertarget{coercion}{%
\subsection{2.7 Coercion}\label{coercion}}

In general, coercion is an attempt by R to be flexible with data types.
When an entry does not match the expected, some of the prebuilt R
functions try to guess what was meant before throwing an error. This can
also lead to confusion. Failing to understand coercion can drive
programmers crazy when attempting to code in R since it behaves quite
differently from most other languages in this regard. Let's learn about
it with some examples.

We said that vectors must be all of the same type. So if we try to
combine, say, numbers and characters, you might expect an error:

\begin{Shaded}
\begin{Highlighting}[]
\NormalTok{x }\OtherTok{\textless{}{-}} \FunctionTok{c}\NormalTok{(}\DecValTok{1}\NormalTok{, }\StringTok{"canada"}\NormalTok{, }\DecValTok{3}\NormalTok{)}
\end{Highlighting}
\end{Shaded}

But we don't get one, not even a warning! What happened? Look at x and
its class:

\begin{Shaded}
\begin{Highlighting}[]
\NormalTok{x}
\end{Highlighting}
\end{Shaded}

\begin{verbatim}
## [1] "1"      "canada" "3"
\end{verbatim}

\begin{Shaded}
\begin{Highlighting}[]
\FunctionTok{class}\NormalTok{(x)}
\end{Highlighting}
\end{Shaded}

\begin{verbatim}
## [1] "character"
\end{verbatim}

R coerced the data into characters. It guessed that because you put a
character string in the vector, you meant the 1 and 3 to actually be
character strings ``1'' and ``3.'' The fact that not even a warning is
issued is an example of how coercion can cause many unnoticed errors in
R.

R also offers functions to change from one type to another. For example,
you can turn numbers into characters with:

\begin{Shaded}
\begin{Highlighting}[]
\NormalTok{x }\OtherTok{\textless{}{-}} \DecValTok{1}\SpecialCharTok{:}\DecValTok{5}
\NormalTok{y }\OtherTok{\textless{}{-}} \FunctionTok{as.character}\NormalTok{(x)}
\NormalTok{y}
\end{Highlighting}
\end{Shaded}

\begin{verbatim}
## [1] "1" "2" "3" "4" "5"
\end{verbatim}

You can turn it back with as.numeric:

\begin{Shaded}
\begin{Highlighting}[]
\FunctionTok{as.numeric}\NormalTok{(y)}
\end{Highlighting}
\end{Shaded}

\begin{verbatim}
## [1] 1 2 3 4 5
\end{verbatim}

This function is actually quite useful since datasets that include
numbers as character strings are common.

\hypertarget{not-availables-na}{%
\subsubsection{2.7.1 Not availables (NA)}\label{not-availables-na}}

When a function tries to coerce one type to another and encounters an
impossible case, it usually gives us a warning and turns the entry into
a special value called an NA for ``not available.'' For example:

\begin{Shaded}
\begin{Highlighting}[]
\NormalTok{x }\OtherTok{\textless{}{-}} \FunctionTok{c}\NormalTok{(}\StringTok{"1"}\NormalTok{, }\StringTok{"b"}\NormalTok{, }\StringTok{"3"}\NormalTok{)}
\FunctionTok{as.numeric}\NormalTok{(x)}
\end{Highlighting}
\end{Shaded}

\begin{verbatim}
## Warning: 강제형변환에 의해 생성된 NA 입니다
\end{verbatim}

\begin{verbatim}
## [1]  1 NA  3
\end{verbatim}

R does not have any guesses for what number you want when you type b, so
it does not try.

As a data scientist you will encounter the NAs often as they are
generally used for missing data, a common problem in real-world
datasets.

\hypertarget{exercises-2}{%
\subsection{2.8 Exercises}\label{exercises-2}}

\begin{enumerate}
\def\labelenumi{\arabic{enumi}.}
\tightlist
\item
  Use the function c to create a vector with the average high
  temperatures in January for Beijing, Lagos, Paris, Rio de Janeiro, San
  Juan, and Toronto, which are 35, 88, 42, 84, 81, and 30 degrees
  Fahrenheit. Call the object temp.
\end{enumerate}

\begin{Shaded}
\begin{Highlighting}[]
\NormalTok{temp }\OtherTok{\textless{}{-}} \FunctionTok{c}\NormalTok{(}\DecValTok{35}\NormalTok{,}\DecValTok{88}\NormalTok{,}\DecValTok{42}\NormalTok{,}\DecValTok{84}\NormalTok{,}\DecValTok{81}\NormalTok{,}\DecValTok{30}\NormalTok{)}
\end{Highlighting}
\end{Shaded}

\begin{enumerate}
\def\labelenumi{\arabic{enumi}.}
\setcounter{enumi}{1}
\tightlist
\item
  Now create a vector with the city names and call the object city.
\end{enumerate}

\begin{Shaded}
\begin{Highlighting}[]
\NormalTok{city }\OtherTok{\textless{}{-}} \FunctionTok{c}\NormalTok{(}\StringTok{"Beijing"}\NormalTok{,}\StringTok{"Lagos"}\NormalTok{,}\StringTok{"Paris"}\NormalTok{,}\StringTok{"Rio de Janeiro"}\NormalTok{,}\StringTok{"San Juan"}\NormalTok{,}\StringTok{"Toronto"}\NormalTok{)}
\end{Highlighting}
\end{Shaded}

\begin{enumerate}
\def\labelenumi{\arabic{enumi}.}
\setcounter{enumi}{2}
\tightlist
\item
  Use the names function and the objects defined in the previous
  exercises to associate the temperature data with its corresponding
  city.
\end{enumerate}

\begin{Shaded}
\begin{Highlighting}[]
\FunctionTok{names}\NormalTok{(temp) }\OtherTok{\textless{}{-}}\NormalTok{ city}
\NormalTok{temp}
\end{Highlighting}
\end{Shaded}

\begin{verbatim}
##        Beijing          Lagos          Paris Rio de Janeiro       San Juan 
##             35             88             42             84             81 
##        Toronto 
##             30
\end{verbatim}

\begin{enumerate}
\def\labelenumi{\arabic{enumi}.}
\setcounter{enumi}{3}
\tightlist
\item
  Use the {[} and : operators to access the temperature of the first
  three cities on the list.
\end{enumerate}

\begin{Shaded}
\begin{Highlighting}[]
\NormalTok{temp[}\DecValTok{1}\SpecialCharTok{:}\DecValTok{3}\NormalTok{]}
\end{Highlighting}
\end{Shaded}

\begin{verbatim}
## Beijing   Lagos   Paris 
##      35      88      42
\end{verbatim}

\begin{enumerate}
\def\labelenumi{\arabic{enumi}.}
\setcounter{enumi}{4}
\tightlist
\item
  Use the {[} operator to access the temperature of Paris and San Juan.
\end{enumerate}

\begin{Shaded}
\begin{Highlighting}[]
\NormalTok{temp[}\FunctionTok{c}\NormalTok{(}\DecValTok{3}\NormalTok{,}\DecValTok{5}\NormalTok{)]}
\end{Highlighting}
\end{Shaded}

\begin{verbatim}
##    Paris San Juan 
##       42       81
\end{verbatim}

\begin{enumerate}
\def\labelenumi{\arabic{enumi}.}
\setcounter{enumi}{5}
\tightlist
\item
  Use the : operator to create a sequence of numbers 12,13,14,\ldots,73.
\end{enumerate}

\begin{Shaded}
\begin{Highlighting}[]
\DecValTok{12}\SpecialCharTok{:}\DecValTok{73}
\end{Highlighting}
\end{Shaded}

\begin{verbatim}
##  [1] 12 13 14 15 16 17 18 19 20 21 22 23 24 25 26 27 28 29 30 31 32 33 34 35 36
## [26] 37 38 39 40 41 42 43 44 45 46 47 48 49 50 51 52 53 54 55 56 57 58 59 60 61
## [51] 62 63 64 65 66 67 68 69 70 71 72 73
\end{verbatim}

\begin{enumerate}
\def\labelenumi{\arabic{enumi}.}
\setcounter{enumi}{6}
\tightlist
\item
  Create a vector containing all the positive odd numbers smaller than
  100.
\end{enumerate}

\begin{Shaded}
\begin{Highlighting}[]
\FunctionTok{seq}\NormalTok{(}\DecValTok{1}\NormalTok{,}\DecValTok{99}\NormalTok{,}\DecValTok{2}\NormalTok{)}
\end{Highlighting}
\end{Shaded}

\begin{verbatim}
##  [1]  1  3  5  7  9 11 13 15 17 19 21 23 25 27 29 31 33 35 37 39 41 43 45 47 49
## [26] 51 53 55 57 59 61 63 65 67 69 71 73 75 77 79 81 83 85 87 89 91 93 95 97 99
\end{verbatim}

\begin{enumerate}
\def\labelenumi{\arabic{enumi}.}
\setcounter{enumi}{7}
\tightlist
\item
  Create a vector of numbers that starts at 6, does not pass 55, and
  adds numbers in increments of 4/7: 6, 6 + 4/7, 6 + 8/7, and so on. How
  many numbers does the list have? Hint: use seq and length.
\end{enumerate}

\begin{Shaded}
\begin{Highlighting}[]
\FunctionTok{length}\NormalTok{(}\FunctionTok{seq}\NormalTok{(}\DecValTok{6}\NormalTok{,}\DecValTok{55}\NormalTok{,}\DecValTok{4}\SpecialCharTok{/}\DecValTok{7}\NormalTok{))}
\end{Highlighting}
\end{Shaded}

\begin{verbatim}
## [1] 86
\end{verbatim}

\begin{enumerate}
\def\labelenumi{\arabic{enumi}.}
\setcounter{enumi}{8}
\tightlist
\item
  What is the class of the following object a \textless- seq(1, 10,
  0.5)?
\end{enumerate}

\begin{Shaded}
\begin{Highlighting}[]
\NormalTok{a }\OtherTok{\textless{}{-}} \FunctionTok{seq}\NormalTok{(}\DecValTok{1}\NormalTok{, }\DecValTok{10}\NormalTok{, }\FloatTok{0.5}\NormalTok{)}
\FunctionTok{class}\NormalTok{(a)}
\end{Highlighting}
\end{Shaded}

\begin{verbatim}
## [1] "numeric"
\end{verbatim}

\begin{enumerate}
\def\labelenumi{\arabic{enumi}.}
\setcounter{enumi}{9}
\tightlist
\item
  What is the class of the following object a \textless- seq(1, 10)?
\end{enumerate}

\begin{Shaded}
\begin{Highlighting}[]
\NormalTok{a }\OtherTok{\textless{}{-}} \FunctionTok{seq}\NormalTok{(}\DecValTok{1}\NormalTok{, }\DecValTok{10}\NormalTok{)}
\FunctionTok{class}\NormalTok{(a)}
\end{Highlighting}
\end{Shaded}

\begin{verbatim}
## [1] "integer"
\end{verbatim}

\begin{enumerate}
\def\labelenumi{\arabic{enumi}.}
\setcounter{enumi}{10}
\tightlist
\item
  The class of class(a\textless-1) is numeric, not integer. R defaults
  to numeric and to force an integer, you need to add the letter L.
  Confirm that the class of 1L is integer.
\end{enumerate}

\begin{Shaded}
\begin{Highlighting}[]
\FunctionTok{class}\NormalTok{(a}\OtherTok{\textless{}{-}}\DecValTok{1}\NormalTok{)}
\end{Highlighting}
\end{Shaded}

\begin{verbatim}
## [1] "numeric"
\end{verbatim}

\begin{Shaded}
\begin{Highlighting}[]
\FunctionTok{class}\NormalTok{(1L)}
\end{Highlighting}
\end{Shaded}

\begin{verbatim}
## [1] "integer"
\end{verbatim}

\begin{enumerate}
\def\labelenumi{\arabic{enumi}.}
\setcounter{enumi}{11}
\tightlist
\item
  Define the following vector:
\end{enumerate}

\begin{Shaded}
\begin{Highlighting}[]
\NormalTok{x }\OtherTok{\textless{}{-}} \FunctionTok{c}\NormalTok{(}\StringTok{"1"}\NormalTok{, }\StringTok{"3"}\NormalTok{, }\StringTok{"5"}\NormalTok{)}
\end{Highlighting}
\end{Shaded}

and coerce it to get integers.

\begin{Shaded}
\begin{Highlighting}[]
\FunctionTok{as.integer}\NormalTok{(x)}
\end{Highlighting}
\end{Shaded}

\begin{verbatim}
## [1] 1 3 5
\end{verbatim}

\begin{Shaded}
\begin{Highlighting}[]
\FunctionTok{class}\NormalTok{(x)}
\end{Highlighting}
\end{Shaded}

\begin{verbatim}
## [1] "character"
\end{verbatim}

\begin{Shaded}
\begin{Highlighting}[]
\FunctionTok{class}\NormalTok{(}\FunctionTok{as.integer}\NormalTok{(x))}
\end{Highlighting}
\end{Shaded}

\begin{verbatim}
## [1] "integer"
\end{verbatim}

\hypertarget{sorting}{%
\subsection{2.9 Sorting}\label{sorting}}

Now that we have mastered some basic R knowledge, let's try to gain some
insights into the safety of different states in the context of gun
murders.

\hypertarget{sort}{%
\subsubsection{2.9.1 sort}\label{sort}}

Say we want to rank the states from least to most gun murders. The
function sort sorts a vector in increasing order. We can therefore see
the largest number of gun murders by typing:

\begin{Shaded}
\begin{Highlighting}[]
\FunctionTok{library}\NormalTok{(dslabs)}
\FunctionTok{data}\NormalTok{(murders)}
\FunctionTok{sort}\NormalTok{(murders}\SpecialCharTok{$}\NormalTok{total)}
\end{Highlighting}
\end{Shaded}

\begin{verbatim}
##  [1]    2    4    5    5    7    8   11   12   12   16   19   21   22   27   32
## [16]   36   38   53   63   65   67   84   93   93   97   97   99  111  116  118
## [31]  120  135  142  207  219  232  246  250  286  293  310  321  351  364  376
## [46]  413  457  517  669  805 1257
\end{verbatim}

However, this does not give us information about which states have which
murder totals. For example, we don't know which state had 1257.

\hypertarget{order}{%
\subsubsection{2.9.2 order}\label{order}}

The function order is closer to what we want. It takes a vector as input
and returns the vector of indexes that sorts the input vector. This may
sound confusing so let's look at a simple example. We can create a
vector and sort it:

\begin{Shaded}
\begin{Highlighting}[]
\NormalTok{x }\OtherTok{\textless{}{-}} \FunctionTok{c}\NormalTok{(}\DecValTok{31}\NormalTok{, }\DecValTok{4}\NormalTok{, }\DecValTok{15}\NormalTok{, }\DecValTok{92}\NormalTok{, }\DecValTok{65}\NormalTok{)}
\FunctionTok{sort}\NormalTok{(x)}
\end{Highlighting}
\end{Shaded}

\begin{verbatim}
## [1]  4 15 31 65 92
\end{verbatim}

Rather than sort the input vector, the function order returns the index
that sorts input vector:

\begin{Shaded}
\begin{Highlighting}[]
\NormalTok{index }\OtherTok{\textless{}{-}} \FunctionTok{order}\NormalTok{(x)}
\NormalTok{x[index]}
\end{Highlighting}
\end{Shaded}

\begin{verbatim}
## [1]  4 15 31 65 92
\end{verbatim}

This is the same output as that returned by sort(x). If we look at this
index, we see why it works:

\begin{Shaded}
\begin{Highlighting}[]
\NormalTok{x}
\end{Highlighting}
\end{Shaded}

\begin{verbatim}
## [1] 31  4 15 92 65
\end{verbatim}

\begin{Shaded}
\begin{Highlighting}[]
\FunctionTok{order}\NormalTok{(x)}
\end{Highlighting}
\end{Shaded}

\begin{verbatim}
## [1] 2 3 1 5 4
\end{verbatim}

The second entry of x is the smallest, so order(x) starts with 2. The
next smallest is the third entry, so the second entry is 3 and so on.

How does this help us order the states by murders? First, remember that
the entries of vectors you access with \$ follow the same order as the
rows in the table. For example, these two vectors containing state names
and abbreviations, respectively, are matched by their order:

\begin{Shaded}
\begin{Highlighting}[]
\NormalTok{murders}\SpecialCharTok{$}\NormalTok{state[}\DecValTok{1}\SpecialCharTok{:}\DecValTok{6}\NormalTok{]}
\end{Highlighting}
\end{Shaded}

\begin{verbatim}
## [1] "Alabama"    "Alaska"     "Arizona"    "Arkansas"   "California"
## [6] "Colorado"
\end{verbatim}

\begin{Shaded}
\begin{Highlighting}[]
\NormalTok{murders}\SpecialCharTok{$}\NormalTok{abb[}\DecValTok{1}\SpecialCharTok{:}\DecValTok{6}\NormalTok{]}
\end{Highlighting}
\end{Shaded}

\begin{verbatim}
## [1] "AL" "AK" "AZ" "AR" "CA" "CO"
\end{verbatim}

This means we can order the state names by their total murders. We first
obtain the index that orders the vectors according to murder totals and
then index the state names vector:

\begin{Shaded}
\begin{Highlighting}[]
\NormalTok{ind }\OtherTok{\textless{}{-}} \FunctionTok{order}\NormalTok{(murders}\SpecialCharTok{$}\NormalTok{total)}
\NormalTok{murders}\SpecialCharTok{$}\NormalTok{abb[ind]}
\end{Highlighting}
\end{Shaded}

\begin{verbatim}
##  [1] "VT" "ND" "NH" "WY" "HI" "SD" "ME" "ID" "MT" "RI" "AK" "IA" "UT" "WV" "NE"
## [16] "OR" "DE" "MN" "KS" "CO" "NM" "NV" "AR" "WA" "CT" "WI" "DC" "OK" "KY" "MA"
## [31] "MS" "AL" "IN" "SC" "TN" "AZ" "NJ" "VA" "NC" "MD" "OH" "MO" "LA" "IL" "GA"
## [46] "MI" "PA" "NY" "FL" "TX" "CA"
\end{verbatim}

According to the above, California had the most murders.

\hypertarget{max-and-which.max}{%
\subsubsection{2.9.3 max and which.max}\label{max-and-which.max}}

If we are only interested in the entry with the largest value, we can
use max for the value:

\begin{Shaded}
\begin{Highlighting}[]
\FunctionTok{max}\NormalTok{(murders}\SpecialCharTok{$}\NormalTok{total)}
\end{Highlighting}
\end{Shaded}

\begin{verbatim}
## [1] 1257
\end{verbatim}

and which.max for the index of the largest value:

\begin{Shaded}
\begin{Highlighting}[]
\NormalTok{i\_max }\OtherTok{\textless{}{-}} \FunctionTok{which.max}\NormalTok{(murders}\SpecialCharTok{$}\NormalTok{total)}
\NormalTok{murders}\SpecialCharTok{$}\NormalTok{state[i\_max]}
\end{Highlighting}
\end{Shaded}

\begin{verbatim}
## [1] "California"
\end{verbatim}

For the minimum, we can use min and which.min in the same way.

Does this mean California is the most dangerous state? In an upcoming
section, we argue that we should be considering rates instead of totals.
Before doing that, we introduce one last order-related function: rank.

\hypertarget{rank}{%
\subsubsection{2.9.4 rank}\label{rank}}

Although not as frequently used as order and sort, the function rank is
also related to order and can be useful. For any given vector it returns
a vector with the rank of the first entry, second entry, etc., of the
input vector. Here is a simple example:

\begin{Shaded}
\begin{Highlighting}[]
\NormalTok{x }\OtherTok{\textless{}{-}} \FunctionTok{c}\NormalTok{(}\DecValTok{31}\NormalTok{, }\DecValTok{4}\NormalTok{, }\DecValTok{15}\NormalTok{, }\DecValTok{92}\NormalTok{, }\DecValTok{65}\NormalTok{)}
\FunctionTok{rank}\NormalTok{(x)}
\end{Highlighting}
\end{Shaded}

\begin{verbatim}
## [1] 3 1 2 5 4
\end{verbatim}

To summarize, let's look at the results of the three functions we have
introduced:

\hypertarget{beware-of-recycling}{%
\subsubsection{2.9.5 Beware of recycling}\label{beware-of-recycling}}

Another common source of unnoticed errors in R is the use of recycling.
We saw that vectors are added elementwise. So if the vectors don't match
in length, it is natural to assume that we should get an error. But we
don't. Notice what happens:

\begin{Shaded}
\begin{Highlighting}[]
\NormalTok{x }\OtherTok{\textless{}{-}} \FunctionTok{c}\NormalTok{(}\DecValTok{1}\NormalTok{, }\DecValTok{2}\NormalTok{, }\DecValTok{3}\NormalTok{)}
\NormalTok{y }\OtherTok{\textless{}{-}} \FunctionTok{c}\NormalTok{(}\DecValTok{10}\NormalTok{, }\DecValTok{20}\NormalTok{, }\DecValTok{30}\NormalTok{, }\DecValTok{40}\NormalTok{, }\DecValTok{50}\NormalTok{, }\DecValTok{60}\NormalTok{, }\DecValTok{70}\NormalTok{)}
\NormalTok{x}\SpecialCharTok{+}\NormalTok{y}
\end{Highlighting}
\end{Shaded}

\begin{verbatim}
## Warning in x + y: 두 객체의 길이가 서로 배수관계에 있지 않습니다
\end{verbatim}

\begin{verbatim}
## [1] 11 22 33 41 52 63 71
\end{verbatim}

We do get a warning, but no error. For the output, R has recycled the
numbers in x. Notice the last digit of numbers in the output.

\hypertarget{exercises-3}{%
\subsection{2.10 Exercises}\label{exercises-3}}

For these exercises we will use the US murders dataset. Make sure you
load it prior to starting.

\begin{Shaded}
\begin{Highlighting}[]
\FunctionTok{library}\NormalTok{(dslabs)}
\FunctionTok{data}\NormalTok{(}\StringTok{"murders"}\NormalTok{)}
\end{Highlighting}
\end{Shaded}

\begin{enumerate}
\def\labelenumi{\arabic{enumi}.}
\tightlist
\item
  Use the \$ operator to access the population size data and store it as
  the object pop. Then use the sort function to redefine pop so that it
  is sorted. Finally, use the {[} operator to report the smallest
  population size.
\end{enumerate}

\begin{Shaded}
\begin{Highlighting}[]
\FunctionTok{head}\NormalTok{(murders)}
\end{Highlighting}
\end{Shaded}

\begin{verbatim}
##        state abb region population total
## 1    Alabama  AL  South    4779736   135
## 2     Alaska  AK   West     710231    19
## 3    Arizona  AZ   West    6392017   232
## 4   Arkansas  AR  South    2915918    93
## 5 California  CA   West   37253956  1257
## 6   Colorado  CO   West    5029196    65
\end{verbatim}

\begin{Shaded}
\begin{Highlighting}[]
\NormalTok{pop }\OtherTok{\textless{}{-}}\NormalTok{ murders}\SpecialCharTok{$}\NormalTok{population}
\FunctionTok{sort}\NormalTok{(pop)}
\end{Highlighting}
\end{Shaded}

\begin{verbatim}
##  [1]   563626   601723   625741   672591   710231   814180   897934   989415
##  [9]  1052567  1316470  1328361  1360301  1567582  1826341  1852994  2059179
## [17]  2700551  2763885  2853118  2915918  2967297  3046355  3574097  3751351
## [25]  3831074  4339367  4533372  4625364  4779736  5029196  5303925  5686986
## [33]  5773552  5988927  6346105  6392017  6483802  6547629  6724540  8001024
## [41]  8791894  9535483  9883640  9920000 11536504 12702379 12830632 19378102
## [49] 19687653 25145561 37253956
\end{verbatim}

\begin{Shaded}
\begin{Highlighting}[]
\NormalTok{sortpop }\OtherTok{\textless{}{-}} \FunctionTok{sort}\NormalTok{(pop)}
\NormalTok{sortpop[}\DecValTok{1}\NormalTok{]}
\end{Highlighting}
\end{Shaded}

\begin{verbatim}
## [1] 563626
\end{verbatim}

\begin{enumerate}
\def\labelenumi{\arabic{enumi}.}
\setcounter{enumi}{1}
\tightlist
\item
  Now instead of the smallest population size, find the index of the
  entry with the smallest population size. Hint: use order instead of
  sort.
\end{enumerate}

\begin{Shaded}
\begin{Highlighting}[]
\FunctionTok{order}\NormalTok{(pop)}
\end{Highlighting}
\end{Shaded}

\begin{verbatim}
##  [1] 51  9 46 35  2 42  8 27 40 30 20 12 13 28 49 32 29 45 17  4 25 16  7 37 38
## [26] 18 19 41  1  6 24 50 21 26 43  3 15 22 48 47 31 34 23 11 36 39 14 33 10 44
## [51]  5
\end{verbatim}

\begin{Shaded}
\begin{Highlighting}[]
\NormalTok{pop[}\DecValTok{51}\NormalTok{]}
\end{Highlighting}
\end{Shaded}

\begin{verbatim}
## [1] 563626
\end{verbatim}

\begin{enumerate}
\def\labelenumi{\arabic{enumi}.}
\setcounter{enumi}{2}
\tightlist
\item
  We can actually perform the same operation as in the previous exercise
  using the function which.min. Write one line of code that does this.
\end{enumerate}

\begin{Shaded}
\begin{Highlighting}[]
\FunctionTok{which.min}\NormalTok{(pop)}
\end{Highlighting}
\end{Shaded}

\begin{verbatim}
## [1] 51
\end{verbatim}

\begin{enumerate}
\def\labelenumi{\arabic{enumi}.}
\setcounter{enumi}{3}
\tightlist
\item
  Now we know how small the smallest state is and we know which row
  represents it. Which state is it? Define a variable states to be the
  state names from the murders data frame. Report the name of the state
  with the smallest population.
\end{enumerate}

\begin{Shaded}
\begin{Highlighting}[]
\NormalTok{states }\OtherTok{\textless{}{-}}\NormalTok{ murders}\SpecialCharTok{$}\NormalTok{state}
\NormalTok{states[}\DecValTok{51}\NormalTok{]}
\end{Highlighting}
\end{Shaded}

\begin{verbatim}
## [1] "Wyoming"
\end{verbatim}

\begin{enumerate}
\def\labelenumi{\arabic{enumi}.}
\setcounter{enumi}{4}
\tightlist
\item
  You can create a data frame using the data.frame function. Here is a
  quick example:
\end{enumerate}

\begin{Shaded}
\begin{Highlighting}[]
\NormalTok{temp }\OtherTok{\textless{}{-}} \FunctionTok{c}\NormalTok{(}\DecValTok{35}\NormalTok{, }\DecValTok{88}\NormalTok{, }\DecValTok{42}\NormalTok{, }\DecValTok{84}\NormalTok{, }\DecValTok{81}\NormalTok{, }\DecValTok{30}\NormalTok{)}
\NormalTok{city }\OtherTok{\textless{}{-}} \FunctionTok{c}\NormalTok{(}\StringTok{"Beijing"}\NormalTok{, }\StringTok{"Lagos"}\NormalTok{, }\StringTok{"Paris"}\NormalTok{, }\StringTok{"Rio de Janeiro"}\NormalTok{, }
          \StringTok{"San Juan"}\NormalTok{, }\StringTok{"Toronto"}\NormalTok{)}
\NormalTok{city\_temps }\OtherTok{\textless{}{-}} \FunctionTok{data.frame}\NormalTok{(}\AttributeTok{name =}\NormalTok{ city, }\AttributeTok{temperature =}\NormalTok{ temp)}
\end{Highlighting}
\end{Shaded}

Use the rank function to determine the population rank of each state
from smallest population size to biggest. Save these ranks in an object
called ranks, then create a data frame with the state name and its rank.
Call the data frame my\_df.

\begin{Shaded}
\begin{Highlighting}[]
\NormalTok{ranks }\OtherTok{\textless{}{-}} \FunctionTok{rank}\NormalTok{(pop)}
\NormalTok{ranks}
\end{Highlighting}
\end{Shaded}

\begin{verbatim}
##  [1] 29  5 36 20 51 30 23  7  2 49 44 12 13 47 37 22 19 26 27 11 33 38 43 31 21
## [26] 34  8 14 17 10 41 16 48 42  4 45 24 25 46  9 28  6 35 50 18  3 40 39 15 32
## [51]  1
\end{verbatim}

\begin{Shaded}
\begin{Highlighting}[]
\NormalTok{my\_df }\OtherTok{\textless{}{-}} \FunctionTok{data.frame}\NormalTok{(}\AttributeTok{name=}\NormalTok{states,}\AttributeTok{rank=}\NormalTok{ranks)}
\NormalTok{my\_df}
\end{Highlighting}
\end{Shaded}

\begin{verbatim}
##                    name rank
## 1               Alabama   29
## 2                Alaska    5
## 3               Arizona   36
## 4              Arkansas   20
## 5            California   51
## 6              Colorado   30
## 7           Connecticut   23
## 8              Delaware    7
## 9  District of Columbia    2
## 10              Florida   49
## 11              Georgia   44
## 12               Hawaii   12
## 13                Idaho   13
## 14             Illinois   47
## 15              Indiana   37
## 16                 Iowa   22
## 17               Kansas   19
## 18             Kentucky   26
## 19            Louisiana   27
## 20                Maine   11
## 21             Maryland   33
## 22        Massachusetts   38
## 23             Michigan   43
## 24            Minnesota   31
## 25          Mississippi   21
## 26             Missouri   34
## 27              Montana    8
## 28             Nebraska   14
## 29               Nevada   17
## 30        New Hampshire   10
## 31           New Jersey   41
## 32           New Mexico   16
## 33             New York   48
## 34       North Carolina   42
## 35         North Dakota    4
## 36                 Ohio   45
## 37             Oklahoma   24
## 38               Oregon   25
## 39         Pennsylvania   46
## 40         Rhode Island    9
## 41       South Carolina   28
## 42         South Dakota    6
## 43            Tennessee   35
## 44                Texas   50
## 45                 Utah   18
## 46              Vermont    3
## 47             Virginia   40
## 48           Washington   39
## 49        West Virginia   15
## 50            Wisconsin   32
## 51              Wyoming    1
\end{verbatim}

\begin{enumerate}
\def\labelenumi{\arabic{enumi}.}
\setcounter{enumi}{5}
\tightlist
\item
  Repeat the previous exercise, but this time order my\_df so that the
  states are ordered from least populous to most populous. Hint: create
  an object ind that stores the indexes needed to order the population
  values. Then use the bracket operator {[} to re-order each column in
  the data frame.
\end{enumerate}

\begin{Shaded}
\begin{Highlighting}[]
\NormalTok{ind }\OtherTok{\textless{}{-}} \FunctionTok{order}\NormalTok{(pop)}
\NormalTok{states[ind]}
\end{Highlighting}
\end{Shaded}

\begin{verbatim}
##  [1] "Wyoming"              "District of Columbia" "Vermont"             
##  [4] "North Dakota"         "Alaska"               "South Dakota"        
##  [7] "Delaware"             "Montana"              "Rhode Island"        
## [10] "New Hampshire"        "Maine"                "Hawaii"              
## [13] "Idaho"                "Nebraska"             "West Virginia"       
## [16] "New Mexico"           "Nevada"               "Utah"                
## [19] "Kansas"               "Arkansas"             "Mississippi"         
## [22] "Iowa"                 "Connecticut"          "Oklahoma"            
## [25] "Oregon"               "Kentucky"             "Louisiana"           
## [28] "South Carolina"       "Alabama"              "Colorado"            
## [31] "Minnesota"            "Wisconsin"            "Maryland"            
## [34] "Missouri"             "Tennessee"            "Arizona"             
## [37] "Indiana"              "Massachusetts"        "Washington"          
## [40] "Virginia"             "New Jersey"           "North Carolina"      
## [43] "Michigan"             "Georgia"              "Ohio"                
## [46] "Pennsylvania"         "Illinois"             "New York"            
## [49] "Florida"              "Texas"                "California"
\end{verbatim}

\begin{Shaded}
\begin{Highlighting}[]
\NormalTok{my\_df }\OtherTok{\textless{}{-}} \FunctionTok{data.frame}\NormalTok{(}\AttributeTok{name=}\NormalTok{states[ind],}\AttributeTok{rank=}\FunctionTok{sort}\NormalTok{(}\FunctionTok{rank}\NormalTok{(pop)))}
\NormalTok{my\_df}
\end{Highlighting}
\end{Shaded}

\begin{verbatim}
##                    name rank
## 1               Wyoming    1
## 2  District of Columbia    2
## 3               Vermont    3
## 4          North Dakota    4
## 5                Alaska    5
## 6          South Dakota    6
## 7              Delaware    7
## 8               Montana    8
## 9          Rhode Island    9
## 10        New Hampshire   10
## 11                Maine   11
## 12               Hawaii   12
## 13                Idaho   13
## 14             Nebraska   14
## 15        West Virginia   15
## 16           New Mexico   16
## 17               Nevada   17
## 18                 Utah   18
## 19               Kansas   19
## 20             Arkansas   20
## 21          Mississippi   21
## 22                 Iowa   22
## 23          Connecticut   23
## 24             Oklahoma   24
## 25               Oregon   25
## 26             Kentucky   26
## 27            Louisiana   27
## 28       South Carolina   28
## 29              Alabama   29
## 30             Colorado   30
## 31            Minnesota   31
## 32            Wisconsin   32
## 33             Maryland   33
## 34             Missouri   34
## 35            Tennessee   35
## 36              Arizona   36
## 37              Indiana   37
## 38        Massachusetts   38
## 39           Washington   39
## 40             Virginia   40
## 41           New Jersey   41
## 42       North Carolina   42
## 43             Michigan   43
## 44              Georgia   44
## 45                 Ohio   45
## 46         Pennsylvania   46
## 47             Illinois   47
## 48             New York   48
## 49              Florida   49
## 50                Texas   50
## 51           California   51
\end{verbatim}

\begin{enumerate}
\def\labelenumi{\arabic{enumi}.}
\setcounter{enumi}{6}
\tightlist
\item
  The na\_example vector represents a series of counts. You can quickly
  examine the object using:
\end{enumerate}

\begin{Shaded}
\begin{Highlighting}[]
\FunctionTok{data}\NormalTok{(}\StringTok{"na\_example"}\NormalTok{)  }
\FunctionTok{str}\NormalTok{(na\_example)}
\end{Highlighting}
\end{Shaded}

\begin{verbatim}
##  int [1:1000] 2 1 3 2 1 3 1 4 3 2 ...
\end{verbatim}

However, when we compute the average with the function mean, we obtain
an NA:

\begin{Shaded}
\begin{Highlighting}[]
\FunctionTok{mean}\NormalTok{(na\_example)}
\end{Highlighting}
\end{Shaded}

\begin{verbatim}
## [1] NA
\end{verbatim}

The is.na function returns a logical vector that tells us which entries
are NA. Assign this logical vector to an object called ind and determine
how many NAs does na\_example have.

\begin{Shaded}
\begin{Highlighting}[]
\NormalTok{ind }\OtherTok{\textless{}{-}} \FunctionTok{is.na}\NormalTok{(na\_example)}
\FunctionTok{sum}\NormalTok{(ind)}
\end{Highlighting}
\end{Shaded}

\begin{verbatim}
## [1] 145
\end{verbatim}

\begin{enumerate}
\def\labelenumi{\arabic{enumi}.}
\setcounter{enumi}{7}
\tightlist
\item
  Now compute the average again, but only for the entries that are not
  NA. Hint: remember the ! operator.
\end{enumerate}

\begin{Shaded}
\begin{Highlighting}[]
\NormalTok{ind }\OtherTok{\textless{}{-}} \FunctionTok{which}\NormalTok{(}\FunctionTok{is.na}\NormalTok{(na\_example)}\SpecialCharTok{!=}\ConstantTok{TRUE}\NormalTok{)}
\FunctionTok{mean}\NormalTok{(na\_example[ind])}
\end{Highlighting}
\end{Shaded}

\begin{verbatim}
## [1] 2.301754
\end{verbatim}

\begin{Shaded}
\begin{Highlighting}[]
\FunctionTok{mean}\NormalTok{(na\_example,}\AttributeTok{na.rm=}\ConstantTok{TRUE}\NormalTok{)}
\end{Highlighting}
\end{Shaded}

\begin{verbatim}
## [1] 2.301754
\end{verbatim}

\hypertarget{vector-arithmetics}{%
\subsection{2.11 Vector arithmetics}\label{vector-arithmetics}}

California had the most murders, but does this mean it is the most
dangerous state? What if it just has many more people than any other
state? We can quickly confirm that California indeed has the largest
population:

\begin{Shaded}
\begin{Highlighting}[]
\FunctionTok{library}\NormalTok{(dslabs)}
\FunctionTok{data}\NormalTok{(}\StringTok{"murders"}\NormalTok{)}
\NormalTok{murders}\SpecialCharTok{$}\NormalTok{state[}\FunctionTok{which.max}\NormalTok{(murders}\SpecialCharTok{$}\NormalTok{population)]}
\end{Highlighting}
\end{Shaded}

\begin{verbatim}
## [1] "California"
\end{verbatim}

with over 37 million inhabitants. It is therefore unfair to compare the
totals if we are interested in learning how safe the state is. What we
really should be computing is the murders per capita. The reports we
describe in the motivating section used murders per 100,000 as the unit.
To compute this quantity, the powerful vector arithmetic capabilities of
R come in handy.

\hypertarget{rescaling-a-vector}{%
\subsubsection{2.11.1 Rescaling a vector}\label{rescaling-a-vector}}

In R, arithmetic operations on vectors occur element-wise. For a quick
example, suppose we have height in inches:

\begin{Shaded}
\begin{Highlighting}[]
\NormalTok{inches }\OtherTok{\textless{}{-}} \FunctionTok{c}\NormalTok{(}\DecValTok{69}\NormalTok{, }\DecValTok{62}\NormalTok{, }\DecValTok{66}\NormalTok{, }\DecValTok{70}\NormalTok{, }\DecValTok{70}\NormalTok{, }\DecValTok{73}\NormalTok{, }\DecValTok{67}\NormalTok{, }\DecValTok{73}\NormalTok{, }\DecValTok{67}\NormalTok{, }\DecValTok{70}\NormalTok{)}
\end{Highlighting}
\end{Shaded}

and want to convert to centimeters. Notice what happens when we multiply
inches by 2.54:

\begin{Shaded}
\begin{Highlighting}[]
\NormalTok{inches }\SpecialCharTok{*} \FloatTok{2.54}
\end{Highlighting}
\end{Shaded}

\begin{verbatim}
##  [1] 175.26 157.48 167.64 177.80 177.80 185.42 170.18 185.42 170.18 177.80
\end{verbatim}

In the line above, we multiplied each element by 2.54. Similarly, if for
each entry we want to compute how many inches taller or shorter than 69
inches, the average height for males, we can subtract it from every
entry like this:

\begin{Shaded}
\begin{Highlighting}[]
\NormalTok{inches }\SpecialCharTok{{-}} \DecValTok{69}
\end{Highlighting}
\end{Shaded}

\begin{verbatim}
##  [1]  0 -7 -3  1  1  4 -2  4 -2  1
\end{verbatim}

\hypertarget{two-vectors}{%
\subsubsection{2.11.2 Two vectors}\label{two-vectors}}

If we have two vectors of the same length, and we sum them in R, they
will be added entry by entry as follows:

The same holds for other mathematical operations, such as -, * and /.

This implies that to compute the murder rates we can simply type:

\begin{Shaded}
\begin{Highlighting}[]
\NormalTok{murder\_rate }\OtherTok{\textless{}{-}}\NormalTok{ murders}\SpecialCharTok{$}\NormalTok{total }\SpecialCharTok{/}\NormalTok{ murders}\SpecialCharTok{$}\NormalTok{population }\SpecialCharTok{*} \DecValTok{100000}
\end{Highlighting}
\end{Shaded}

Once we do this, we notice that California is no longer near the top of
the list. In fact, we can use what we have learned to order the states
by murder rate:

\begin{Shaded}
\begin{Highlighting}[]
\NormalTok{murders}\SpecialCharTok{$}\NormalTok{abb[}\FunctionTok{order}\NormalTok{(murder\_rate)]}
\end{Highlighting}
\end{Shaded}

\begin{verbatim}
##  [1] "VT" "NH" "HI" "ND" "IA" "ID" "UT" "ME" "WY" "OR" "SD" "MN" "MT" "CO" "WA"
## [16] "WV" "RI" "WI" "NE" "MA" "IN" "KS" "NY" "KY" "AK" "OH" "CT" "NJ" "AL" "IL"
## [31] "OK" "NC" "NV" "VA" "AR" "TX" "NM" "CA" "FL" "TN" "PA" "AZ" "GA" "MS" "MI"
## [46] "DE" "SC" "MD" "MO" "LA" "DC"
\end{verbatim}

\hypertarget{exercises-4}{%
\subsection{2.12 Exercises}\label{exercises-4}}

\begin{enumerate}
\def\labelenumi{\arabic{enumi}.}
\tightlist
\item
  Previously we created this data frame:
\end{enumerate}

\begin{Shaded}
\begin{Highlighting}[]
\NormalTok{temp }\OtherTok{\textless{}{-}} \FunctionTok{c}\NormalTok{(}\DecValTok{35}\NormalTok{, }\DecValTok{88}\NormalTok{, }\DecValTok{42}\NormalTok{, }\DecValTok{84}\NormalTok{, }\DecValTok{81}\NormalTok{, }\DecValTok{30}\NormalTok{)}
\NormalTok{city }\OtherTok{\textless{}{-}} \FunctionTok{c}\NormalTok{(}\StringTok{"Beijing"}\NormalTok{, }\StringTok{"Lagos"}\NormalTok{, }\StringTok{"Paris"}\NormalTok{, }\StringTok{"Rio de Janeiro"}\NormalTok{, }
          \StringTok{"San Juan"}\NormalTok{, }\StringTok{"Toronto"}\NormalTok{)}
\NormalTok{city\_temps }\OtherTok{\textless{}{-}} \FunctionTok{data.frame}\NormalTok{(}\AttributeTok{name =}\NormalTok{ city, }\AttributeTok{temperature =}\NormalTok{ temp)}
\end{Highlighting}
\end{Shaded}

Remake the data frame using the code above, but add a line that converts
the temperature from Fahrenheit to Celsius. The conversion is C =
\frac{5}{9} \times (F - 32).

\begin{Shaded}
\begin{Highlighting}[]
\NormalTok{C }\OtherTok{\textless{}{-}} \DecValTok{5}\SpecialCharTok{/}\DecValTok{9} \SpecialCharTok{*}\NormalTok{ (temp}\DecValTok{{-}32}\NormalTok{)}
\NormalTok{city\_temps }\OtherTok{\textless{}{-}} \FunctionTok{data.frame}\NormalTok{(}\AttributeTok{name =}\NormalTok{ city, }\AttributeTok{fahrenheit =}\NormalTok{ temp, }\AttributeTok{celsius =}\NormalTok{ C)}
\NormalTok{city\_temps}
\end{Highlighting}
\end{Shaded}

\begin{verbatim}
##             name fahrenheit   celsius
## 1        Beijing         35  1.666667
## 2          Lagos         88 31.111111
## 3          Paris         42  5.555556
## 4 Rio de Janeiro         84 28.888889
## 5       San Juan         81 27.222222
## 6        Toronto         30 -1.111111
\end{verbatim}

\begin{enumerate}
\def\labelenumi{\arabic{enumi}.}
\setcounter{enumi}{1}
\tightlist
\item
  What is the following sum 1+1/2\^{}2 + 1/3\^{}2 +\ldots+ 1/100\^{}2?
  Hint: thanks to Euler, we know it should be close to pi\^{}2/6.
\end{enumerate}

\begin{Shaded}
\begin{Highlighting}[]
\NormalTok{n }\OtherTok{\textless{}{-}} \DecValTok{1}\SpecialCharTok{:}\DecValTok{100}
\FunctionTok{sum}\NormalTok{(}\DecValTok{1}\SpecialCharTok{/}\NormalTok{n}\SpecialCharTok{\^{}}\DecValTok{2}\NormalTok{)}
\end{Highlighting}
\end{Shaded}

\begin{verbatim}
## [1] 1.634984
\end{verbatim}

\begin{Shaded}
\begin{Highlighting}[]
\NormalTok{pi}\SpecialCharTok{\^{}}\DecValTok{2}\SpecialCharTok{/}\DecValTok{6}
\end{Highlighting}
\end{Shaded}

\begin{verbatim}
## [1] 1.644934
\end{verbatim}

\begin{enumerate}
\def\labelenumi{\arabic{enumi}.}
\setcounter{enumi}{2}
\tightlist
\item
  Compute the per 100,000 murder rate for each state and store it in the
  object murder\_rate. Then compute the average murder rate for the US
  using the function mean. What is the average?
\end{enumerate}

\begin{Shaded}
\begin{Highlighting}[]
\NormalTok{murder\_rate }\OtherTok{\textless{}{-}}\NormalTok{ murders}\SpecialCharTok{$}\NormalTok{total }\SpecialCharTok{/}\NormalTok{ murders}\SpecialCharTok{$}\NormalTok{population }\SpecialCharTok{*} \DecValTok{100000}
\FunctionTok{mean}\NormalTok{(murder\_rate)}
\end{Highlighting}
\end{Shaded}

\begin{verbatim}
## [1] 2.779125
\end{verbatim}

\hypertarget{indexing}{%
\subsection{2.13 Indexing}\label{indexing}}

R provides a powerful and convenient way of indexing vectors. We can,
for example, subset a vector based on properties of another vector. In
this section, we continue working with our US murders example, which we
can load like this:

\begin{Shaded}
\begin{Highlighting}[]
\FunctionTok{library}\NormalTok{(dslabs)}
\FunctionTok{data}\NormalTok{(}\StringTok{"murders"}\NormalTok{)}
\end{Highlighting}
\end{Shaded}

\hypertarget{subsetting-with-logicals}{%
\subsubsection{2.13.1 Subsetting with
logicals}\label{subsetting-with-logicals}}

We have now calculated the murder rate using:

\begin{Shaded}
\begin{Highlighting}[]
\NormalTok{murder\_rate }\OtherTok{\textless{}{-}}\NormalTok{ murders}\SpecialCharTok{$}\NormalTok{total }\SpecialCharTok{/}\NormalTok{ murders}\SpecialCharTok{$}\NormalTok{population }\SpecialCharTok{*} \DecValTok{100000}
\end{Highlighting}
\end{Shaded}

Imagine you are moving from Italy where, according to an ABC news
report, the murder rate is only 0.71 per 100,000. You would prefer to
move to a state with a similar murder rate. Another powerful feature of
R is that we can use logicals to index vectors. If we compare a vector
to a single number, it actually performs the test for each entry. The
following is an example related to the question above:

\begin{Shaded}
\begin{Highlighting}[]
\NormalTok{ind }\OtherTok{\textless{}{-}}\NormalTok{ murder\_rate }\SpecialCharTok{\textless{}} \FloatTok{0.71}
\end{Highlighting}
\end{Shaded}

If we instead want to know if a value is less or equal, we can use:

\begin{Shaded}
\begin{Highlighting}[]
\NormalTok{ind }\OtherTok{\textless{}{-}}\NormalTok{ murder\_rate }\SpecialCharTok{\textless{}=} \FloatTok{0.71}
\end{Highlighting}
\end{Shaded}

Note that we get back a logical vector with TRUE for each entry smaller
than or equal to 0.71. To see which states these are, we can leverage
the fact that vectors can be indexed with logicals.

\begin{Shaded}
\begin{Highlighting}[]
\NormalTok{murders}\SpecialCharTok{$}\NormalTok{state[ind]}
\end{Highlighting}
\end{Shaded}

\begin{verbatim}
## [1] "Hawaii"        "Iowa"          "New Hampshire" "North Dakota" 
## [5] "Vermont"
\end{verbatim}

In order to count how many are TRUE, the function sum returns the sum of
the entries of a vector and logical vectors get coerced to numeric with
TRUE coded as 1 and FALSE as 0. Thus we can count the states using:

\begin{Shaded}
\begin{Highlighting}[]
\FunctionTok{sum}\NormalTok{(ind)}
\end{Highlighting}
\end{Shaded}

\begin{verbatim}
## [1] 5
\end{verbatim}

\hypertarget{logical-operators}{%
\subsubsection{2.13.2 Logical operators}\label{logical-operators}}

Suppose we like the mountains and we want to move to a safe state in the
western region of the country. We want the murder rate to be at most 1.
In this case, we want two different things to be true. Here we can use
the logical operator and, which in R is represented with \&. This
operation results in TRUE only when both logicals are TRUE. To see this,
consider this example:

\begin{Shaded}
\begin{Highlighting}[]
\ConstantTok{TRUE} \SpecialCharTok{\&} \ConstantTok{TRUE}
\end{Highlighting}
\end{Shaded}

\begin{verbatim}
## [1] TRUE
\end{verbatim}

\begin{Shaded}
\begin{Highlighting}[]
\ConstantTok{TRUE} \SpecialCharTok{\&} \ConstantTok{FALSE}
\end{Highlighting}
\end{Shaded}

\begin{verbatim}
## [1] FALSE
\end{verbatim}

\begin{Shaded}
\begin{Highlighting}[]
\ConstantTok{FALSE} \SpecialCharTok{\&} \ConstantTok{FALSE}
\end{Highlighting}
\end{Shaded}

\begin{verbatim}
## [1] FALSE
\end{verbatim}

For our example, we can form two logicals:

\begin{Shaded}
\begin{Highlighting}[]
\NormalTok{west }\OtherTok{\textless{}{-}}\NormalTok{ murders}\SpecialCharTok{$}\NormalTok{region }\SpecialCharTok{==} \StringTok{"West"}
\NormalTok{safe }\OtherTok{\textless{}{-}}\NormalTok{ murder\_rate }\SpecialCharTok{\textless{}=} \DecValTok{1}
\end{Highlighting}
\end{Shaded}

and we can use the \& to get a vector of logicals that tells us which
states satisfy both conditions:

\begin{Shaded}
\begin{Highlighting}[]
\NormalTok{ind }\OtherTok{\textless{}{-}}\NormalTok{ safe }\SpecialCharTok{\&}\NormalTok{ west}
\NormalTok{murders}\SpecialCharTok{$}\NormalTok{state[ind]}
\end{Highlighting}
\end{Shaded}

\begin{verbatim}
## [1] "Hawaii"  "Idaho"   "Oregon"  "Utah"    "Wyoming"
\end{verbatim}

\hypertarget{which}{%
\subsubsection{2.13.3 which}\label{which}}

Suppose we want to look up California's murder rate. For this type of
operation, it is convenient to convert vectors of logicals into indexes
instead of keeping long vectors of logicals. The function which tells us
which entries of a logical vector are TRUE. So we can type:

\begin{Shaded}
\begin{Highlighting}[]
\NormalTok{ind }\OtherTok{\textless{}{-}} \FunctionTok{which}\NormalTok{(murders}\SpecialCharTok{$}\NormalTok{state }\SpecialCharTok{==} \StringTok{"California"}\NormalTok{)}
\NormalTok{murder\_rate[ind]}
\end{Highlighting}
\end{Shaded}

\begin{verbatim}
## [1] 3.374138
\end{verbatim}

\hypertarget{match}{%
\subsubsection{2.13.4 match}\label{match}}

If instead of just one state we want to find out the murder rates for
several states, say New York, Florida, and Texas, we can use the
function match. This function tells us which indexes of a second vector
match each of the entries of a first vector:

\begin{Shaded}
\begin{Highlighting}[]
\NormalTok{ind }\OtherTok{\textless{}{-}} \FunctionTok{match}\NormalTok{(}\FunctionTok{c}\NormalTok{(}\StringTok{"New York"}\NormalTok{, }\StringTok{"Florida"}\NormalTok{, }\StringTok{"Texas"}\NormalTok{), murders}\SpecialCharTok{$}\NormalTok{state)}
\NormalTok{ind}
\end{Highlighting}
\end{Shaded}

\begin{verbatim}
## [1] 33 10 44
\end{verbatim}

Now we can look at the murder rates:

\begin{Shaded}
\begin{Highlighting}[]
\NormalTok{murder\_rate[ind]}
\end{Highlighting}
\end{Shaded}

\begin{verbatim}
## [1] 2.667960 3.398069 3.201360
\end{verbatim}

\begin{Shaded}
\begin{Highlighting}[]
\CommentTok{\#\textgreater{} [1] 2.67 3.40 3.20}
\end{Highlighting}
\end{Shaded}

\hypertarget{in}{%
\subsubsection{2.13.5 \%in\%}\label{in}}

If rather than an index we want a logical that tells us whether or not
each element of a first vector is in a second, we can use the function
\%in\%. Let's imagine you are not sure if Boston, Dakota, and Washington
are states. You can find out like this:

\begin{Shaded}
\begin{Highlighting}[]
\FunctionTok{c}\NormalTok{(}\StringTok{"Boston"}\NormalTok{, }\StringTok{"Dakota"}\NormalTok{, }\StringTok{"Washington"}\NormalTok{) }\SpecialCharTok{\%in\%}\NormalTok{ murders}\SpecialCharTok{$}\NormalTok{state}
\end{Highlighting}
\end{Shaded}

\begin{verbatim}
## [1] FALSE FALSE  TRUE
\end{verbatim}

Note that we will be using \%in\% often throughout the book.

Advanced: There is a connection between match and \%in\% through which.
To see this, notice that the following two lines produce the same index
(although in different order):

\begin{Shaded}
\begin{Highlighting}[]
\FunctionTok{match}\NormalTok{(}\FunctionTok{c}\NormalTok{(}\StringTok{"New York"}\NormalTok{, }\StringTok{"Florida"}\NormalTok{, }\StringTok{"Texas"}\NormalTok{), murders}\SpecialCharTok{$}\NormalTok{state)}
\end{Highlighting}
\end{Shaded}

\begin{verbatim}
## [1] 33 10 44
\end{verbatim}

\begin{Shaded}
\begin{Highlighting}[]
\FunctionTok{which}\NormalTok{(murders}\SpecialCharTok{$}\NormalTok{state}\SpecialCharTok{\%in\%}\FunctionTok{c}\NormalTok{(}\StringTok{"New York"}\NormalTok{, }\StringTok{"Florida"}\NormalTok{, }\StringTok{"Texas"}\NormalTok{))}
\end{Highlighting}
\end{Shaded}

\begin{verbatim}
## [1] 10 33 44
\end{verbatim}

\hypertarget{exercises-5}{%
\subsection{2.14 Exercises}\label{exercises-5}}

Start by loading the library and data.

\begin{Shaded}
\begin{Highlighting}[]
\FunctionTok{library}\NormalTok{(dslabs)}
\FunctionTok{data}\NormalTok{(murders)}
\end{Highlighting}
\end{Shaded}

\begin{enumerate}
\def\labelenumi{\arabic{enumi}.}
\tightlist
\item
  Compute the per 100,000 murder rate for each state and store it in an
  object called murder\_rate. Then use logical operators to create a
  logical vector named low that tells us which entries of murder\_rate
  are lower than 1.
\end{enumerate}

\begin{Shaded}
\begin{Highlighting}[]
\NormalTok{murder\_rate }\OtherTok{\textless{}{-}}\NormalTok{ murders}\SpecialCharTok{$}\NormalTok{total }\SpecialCharTok{/}\NormalTok{ murders}\SpecialCharTok{$}\NormalTok{population }\SpecialCharTok{*} \DecValTok{100000}
\NormalTok{low }\OtherTok{\textless{}{-}}\NormalTok{ murder\_rate }\SpecialCharTok{\textless{}} \DecValTok{1}
\NormalTok{low}
\end{Highlighting}
\end{Shaded}

\begin{verbatim}
##  [1] FALSE FALSE FALSE FALSE FALSE FALSE FALSE FALSE FALSE FALSE FALSE  TRUE
## [13]  TRUE FALSE FALSE  TRUE FALSE FALSE FALSE  TRUE FALSE FALSE FALSE  TRUE
## [25] FALSE FALSE FALSE FALSE FALSE  TRUE FALSE FALSE FALSE FALSE  TRUE FALSE
## [37] FALSE  TRUE FALSE FALSE FALSE  TRUE FALSE FALSE  TRUE  TRUE FALSE FALSE
## [49] FALSE FALSE  TRUE
\end{verbatim}

\begin{enumerate}
\def\labelenumi{\arabic{enumi}.}
\setcounter{enumi}{1}
\tightlist
\item
  Now use the results from the previous exercise and the function which
  to determine the indices of murder\_rate associated with values lower
  than 1.
\end{enumerate}

\begin{Shaded}
\begin{Highlighting}[]
\FunctionTok{which}\NormalTok{(low)}
\end{Highlighting}
\end{Shaded}

\begin{verbatim}
##  [1] 12 13 16 20 24 30 35 38 42 45 46 51
\end{verbatim}

\begin{enumerate}
\def\labelenumi{\arabic{enumi}.}
\setcounter{enumi}{2}
\tightlist
\item
  Use the results from the previous exercise to report the names of the
  states with murder rates lower than 1.
\end{enumerate}

\begin{Shaded}
\begin{Highlighting}[]
\NormalTok{murders}\SpecialCharTok{$}\NormalTok{state[low]}
\end{Highlighting}
\end{Shaded}

\begin{verbatim}
##  [1] "Hawaii"        "Idaho"         "Iowa"          "Maine"        
##  [5] "Minnesota"     "New Hampshire" "North Dakota"  "Oregon"       
##  [9] "South Dakota"  "Utah"          "Vermont"       "Wyoming"
\end{verbatim}

\begin{enumerate}
\def\labelenumi{\arabic{enumi}.}
\setcounter{enumi}{3}
\tightlist
\item
  Now extend the code from exercises 2 and 3 to report the states in the
  Northeast with murder rates lower than 1. Hint: use the previously
  defined logical vector low and the logical operator \&.
\end{enumerate}

\begin{Shaded}
\begin{Highlighting}[]
\NormalTok{northeast }\OtherTok{\textless{}{-}}\NormalTok{ murders}\SpecialCharTok{$}\NormalTok{region }\SpecialCharTok{==} \StringTok{"Northeast"}
\NormalTok{ind }\OtherTok{\textless{}{-}}\NormalTok{ low }\SpecialCharTok{\&}\NormalTok{ northeast}
\NormalTok{murders}\SpecialCharTok{$}\NormalTok{state[ind]}
\end{Highlighting}
\end{Shaded}

\begin{verbatim}
## [1] "Maine"         "New Hampshire" "Vermont"
\end{verbatim}

\begin{enumerate}
\def\labelenumi{\arabic{enumi}.}
\setcounter{enumi}{4}
\tightlist
\item
  In a previous exercise we computed the murder rate for each state and
  the average of these numbers. How many states are below the average?
\end{enumerate}

\begin{Shaded}
\begin{Highlighting}[]
\NormalTok{average }\OtherTok{\textless{}{-}} \FunctionTok{mean}\NormalTok{(murder\_rate)}
\FunctionTok{length}\NormalTok{(murder\_rate[murder\_rate }\SpecialCharTok{\textless{}}\NormalTok{ average])}
\end{Highlighting}
\end{Shaded}

\begin{verbatim}
## [1] 27
\end{verbatim}

\begin{enumerate}
\def\labelenumi{\arabic{enumi}.}
\setcounter{enumi}{5}
\tightlist
\item
  Use the match function to identify the states with abbreviations AK,
  MI, and IA. Hint: start by defining an index of the entries of
  murders\$abb that match the three abbreviations, then use the {[}
  operator to extract the states.
\end{enumerate}

\begin{Shaded}
\begin{Highlighting}[]
\NormalTok{ind }\OtherTok{\textless{}{-}} \FunctionTok{match}\NormalTok{(}\FunctionTok{c}\NormalTok{(}\StringTok{"AK"}\NormalTok{,}\StringTok{"MI"}\NormalTok{,}\StringTok{"IA"}\NormalTok{),murders}\SpecialCharTok{$}\NormalTok{abb)}
\NormalTok{murders}\SpecialCharTok{$}\NormalTok{state[ind]}
\end{Highlighting}
\end{Shaded}

\begin{verbatim}
## [1] "Alaska"   "Michigan" "Iowa"
\end{verbatim}

\begin{enumerate}
\def\labelenumi{\arabic{enumi}.}
\setcounter{enumi}{6}
\tightlist
\item
  Use the \%in\% operator to create a logical vector that answers the
  question: which of the following are actual abbreviations: MA, ME, MI,
  MO, MU?
\end{enumerate}

\begin{Shaded}
\begin{Highlighting}[]
\FunctionTok{c}\NormalTok{(}\StringTok{"MA"}\NormalTok{,}\StringTok{"ME"}\NormalTok{,}\StringTok{"MI"}\NormalTok{,}\StringTok{"MO"}\NormalTok{,}\StringTok{"MU"}\NormalTok{) }\SpecialCharTok{\%in\%}\NormalTok{ murders}\SpecialCharTok{$}\NormalTok{abb}
\end{Highlighting}
\end{Shaded}

\begin{verbatim}
## [1]  TRUE  TRUE  TRUE  TRUE FALSE
\end{verbatim}

\begin{enumerate}
\def\labelenumi{\arabic{enumi}.}
\setcounter{enumi}{7}
\tightlist
\item
  Extend the code you used in exercise 7 to report the one entry that is
  not an actual abbreviation. Hint: use the ! operator, which turns
  FALSE into TRUE and vice versa, then which to obtain an index.
\end{enumerate}

\begin{Shaded}
\begin{Highlighting}[]
\FunctionTok{c}\NormalTok{(}\StringTok{"MA"}\NormalTok{,}\StringTok{"ME"}\NormalTok{,}\StringTok{"MI"}\NormalTok{,}\StringTok{"MO"}\NormalTok{,}\StringTok{"MU"}\NormalTok{) }\SpecialCharTok{\%in\%}\NormalTok{ murders}\SpecialCharTok{$}\NormalTok{abb }\SpecialCharTok{!=} \ConstantTok{TRUE}
\end{Highlighting}
\end{Shaded}

\begin{verbatim}
## [1] FALSE FALSE FALSE FALSE  TRUE
\end{verbatim}

\begin{Shaded}
\begin{Highlighting}[]
\FunctionTok{which}\NormalTok{(murders}\SpecialCharTok{$}\NormalTok{abb}\SpecialCharTok{\%in\%}\FunctionTok{c}\NormalTok{(}\StringTok{"MA"}\NormalTok{,}\StringTok{"ME"}\NormalTok{,}\StringTok{"MI"}\NormalTok{,}\StringTok{"MO"}\NormalTok{,}\StringTok{"MU"}\NormalTok{))}
\end{Highlighting}
\end{Shaded}

\begin{verbatim}
## [1] 20 22 23 26
\end{verbatim}

\hypertarget{basic-plots}{%
\subsection{2.15 Basic plots}\label{basic-plots}}

In Chapter 7 we describe an add-on package that provides a powerful
approach to producing plots in R. We then have an entire part on Data
Visualization in which we provide many examples. Here we briefly
describe some of the functions that are available in a basic R
installation.

\hypertarget{plot}{%
\subsubsection{2.15.1 plot}\label{plot}}

The plot function can be used to make scatterplots. Here is a plot of
total murders versus population.

\begin{Shaded}
\begin{Highlighting}[]
\NormalTok{x }\OtherTok{\textless{}{-}}\NormalTok{ murders}\SpecialCharTok{$}\NormalTok{population }\SpecialCharTok{/} \DecValTok{10}\SpecialCharTok{\^{}}\DecValTok{6}
\NormalTok{y }\OtherTok{\textless{}{-}}\NormalTok{ murders}\SpecialCharTok{$}\NormalTok{total}
\FunctionTok{plot}\NormalTok{(x, y)}
\end{Highlighting}
\end{Shaded}

\includegraphics{Chapter-2_files/figure-latex/unnamed-chunk-166-1.pdf}

For a quick plot that avoids accessing variables twice, we can use the
with function:

\begin{Shaded}
\begin{Highlighting}[]
\FunctionTok{with}\NormalTok{(murders, }\FunctionTok{plot}\NormalTok{(population, total))}
\end{Highlighting}
\end{Shaded}

\includegraphics{Chapter-2_files/figure-latex/unnamed-chunk-167-1.pdf}

The function with lets us use the murders column names in the plot
function. It also works with any data frames and any function.

\hypertarget{hist}{%
\subsubsection{2.15.2 hist}\label{hist}}

We will describe histograms as they relate to distributions in the Data
Visualization part of the book. Here we will simply note that histograms
are a powerful graphical summary of a list of numbers that gives you a
general overview of the types of values you have. We can make a
histogram of our murder rates by simply typing:

\begin{Shaded}
\begin{Highlighting}[]
\NormalTok{x }\OtherTok{\textless{}{-}} \FunctionTok{with}\NormalTok{(murders, total }\SpecialCharTok{/}\NormalTok{ population }\SpecialCharTok{*} \DecValTok{100000}\NormalTok{)}
\FunctionTok{hist}\NormalTok{(x)}
\end{Highlighting}
\end{Shaded}

\includegraphics{Chapter-2_files/figure-latex/unnamed-chunk-168-1.pdf}

We can see that there is a wide range of values with most of them
between 2 and 3 and one very extreme case with a murder rate of more
than 15:

\begin{Shaded}
\begin{Highlighting}[]
\NormalTok{murders}\SpecialCharTok{$}\NormalTok{state[}\FunctionTok{which.max}\NormalTok{(x)]}
\end{Highlighting}
\end{Shaded}

\begin{verbatim}
## [1] "District of Columbia"
\end{verbatim}

\hypertarget{boxplot}{%
\subsubsection{2.15.3 boxplot}\label{boxplot}}

Boxplots will also be described in the Data Visualization part of the
book. They provide a more terse summary than histograms, but they are
easier to stack with other boxplots. For example, here we can use them
to compare the different regions:

\begin{Shaded}
\begin{Highlighting}[]
\NormalTok{murders}\SpecialCharTok{$}\NormalTok{rate }\OtherTok{\textless{}{-}} \FunctionTok{with}\NormalTok{(murders, total }\SpecialCharTok{/}\NormalTok{ population }\SpecialCharTok{*} \DecValTok{100000}\NormalTok{)}
\FunctionTok{boxplot}\NormalTok{(rate}\SpecialCharTok{\textasciitilde{}}\NormalTok{region, }\AttributeTok{data =}\NormalTok{ murders)}
\end{Highlighting}
\end{Shaded}

\includegraphics{Chapter-2_files/figure-latex/unnamed-chunk-170-1.pdf}

We can see that the South has higher murder rates than the other three
regions.

\hypertarget{image}{%
\subsubsection{2.15.4 image}\label{image}}

The image function displays the values in a matrix using color. Here is
a quick example:

\begin{Shaded}
\begin{Highlighting}[]
\NormalTok{x }\OtherTok{\textless{}{-}} \FunctionTok{matrix}\NormalTok{(}\DecValTok{1}\SpecialCharTok{:}\DecValTok{120}\NormalTok{, }\DecValTok{12}\NormalTok{, }\DecValTok{10}\NormalTok{)}
\FunctionTok{image}\NormalTok{(x)}
\end{Highlighting}
\end{Shaded}

\includegraphics{Chapter-2_files/figure-latex/unnamed-chunk-171-1.pdf}

\hypertarget{exercises-6}{%
\subsection{2.16 Exercises}\label{exercises-6}}

\begin{enumerate}
\def\labelenumi{\arabic{enumi}.}
\tightlist
\item
  We made a plot of total murders versus population and noted a strong
  relationship. Not surprisingly, states with larger populations had
  more murders.
\end{enumerate}

\begin{Shaded}
\begin{Highlighting}[]
\FunctionTok{library}\NormalTok{(dslabs)}
\FunctionTok{data}\NormalTok{(murders)}
\NormalTok{population\_in\_millions }\OtherTok{\textless{}{-}}\NormalTok{ murders}\SpecialCharTok{$}\NormalTok{population}\SpecialCharTok{/}\DecValTok{10}\SpecialCharTok{\^{}}\DecValTok{6}
\NormalTok{total\_gun\_murders }\OtherTok{\textless{}{-}}\NormalTok{ murders}\SpecialCharTok{$}\NormalTok{total}
\FunctionTok{plot}\NormalTok{(population\_in\_millions, total\_gun\_murders)}
\end{Highlighting}
\end{Shaded}

\includegraphics{Chapter-2_files/figure-latex/unnamed-chunk-172-1.pdf}

Keep in mind that many states have populations below 5 million and are
bunched up. We may gain further insights from making this plot in the
log scale. Transform the variables using the log10 transformation and
then plot them.

\begin{Shaded}
\begin{Highlighting}[]
\NormalTok{population\_in\_log }\OtherTok{\textless{}{-}} \FunctionTok{log}\NormalTok{(murders}\SpecialCharTok{$}\NormalTok{population,}\DecValTok{10}\NormalTok{)}
\FunctionTok{plot}\NormalTok{(population\_in\_log, total\_gun\_murders)}
\end{Highlighting}
\end{Shaded}

\includegraphics{Chapter-2_files/figure-latex/unnamed-chunk-173-1.pdf}

\begin{enumerate}
\def\labelenumi{\arabic{enumi}.}
\setcounter{enumi}{1}
\tightlist
\item
  Create a histogram of the state populations.
\end{enumerate}

\begin{Shaded}
\begin{Highlighting}[]
\NormalTok{x }\OtherTok{\textless{}{-}} \FunctionTok{with}\NormalTok{(murders,population)}
\FunctionTok{hist}\NormalTok{(x)}
\end{Highlighting}
\end{Shaded}

\includegraphics{Chapter-2_files/figure-latex/unnamed-chunk-174-1.pdf}

\begin{enumerate}
\def\labelenumi{\arabic{enumi}.}
\setcounter{enumi}{2}
\tightlist
\item
  Generate boxplots of the state populations by region.
\end{enumerate}

\begin{Shaded}
\begin{Highlighting}[]
\FunctionTok{boxplot}\NormalTok{(population}\SpecialCharTok{\textasciitilde{}}\NormalTok{region,}\AttributeTok{data=}\NormalTok{murders)}
\end{Highlighting}
\end{Shaded}

\includegraphics{Chapter-2_files/figure-latex/unnamed-chunk-175-1.pdf}

\end{document}
